
% -------------------------------------------
% Items to substitute into the ivoatex document template.
%
%\ivoagroup{Data Model Working Group}

%\title{Mango}


%\author{Laurent Michel}
    
%\author{Fran??ois Bonnarel}
    
%\author{Gilles Landais}
    
%\author{Mireille Louys}
    
%\author{Marco Molinaro}
    
%\author{Jesue Salgado}
    
%\previousversion{0.0}
      
% -------------------------------------------

\pagebreak
\section{Model: mango }
  
  % INSERT FIGURE HERE
  %\begin{figure}[h]
  %\begin{center}
  %  \includegraphics[width=\textwidth]{????.png}
  %  \caption{???}\label{fig:????}
  %\end{center}
  %\end{figure}

  Data model based oon components and data association for source data

  \subsection{AssociatedData (Abstract)}
  \label{sect:AssociatedData}
    Abstract reference to a particular dataset associated to the MANGO entity. This class is used to specify the type of the dataset as well as its role.

    \subsubsection{AssociatedData.semantic}
      \textbf{vodml-id: AssociatedData.semantic} \newline
      \textbf{type: \hyperref[sect:VocabularyTerm]{mango:VocabularyTerm}} \newline
      \textbf{multiplicity: 1} \newline 
      Semantic concept giving the nature of the associated data. As long as the vocabulary is not set, the possible values of this attribute are given by the LinkSemantic enumeration.

    \subsubsection{AssociatedData.description}
      \textbf{vodml-id: AssociatedData.description} \newline
      \textbf{type: \hyperref[sect:ivoa]{ivoa:string}} \newline
      \textbf{multiplicity: 1} \newline 
      Free text description of the associated data

  \subsection{AssociatedMangoInstance}
  \label{sect:AssociatedMangoInstance}
    Reference to another MANGO instance that is part of the associated data.

    \subsubsection{AssociatedMangoInstance.mangoInstance}
      \textbf{vodml-id: AssociatedMangoInstance.mangoInstance} \newline
      \textbf{type: \hyperref[sect:MangoObject]{mango:MangoObject}} \newline
      \textbf{multiplicity: 1} \newline 
      Composition link pointing on one MANGO instance associated with the source.

  \subsection{BitField}
  \label{sect:BitField}
    State of a property for which each value is represented by a bit so that several states can be included in the same numerical value. The values defined in the associated StatusValues must correspond to bit patterns. This constraint is not managed by the model.

  \subsection{Color}
  \label{sect:Color}
    Photometric property describing a color of the MANGO source. color is not an intrinsic quantity of the MANGO source but a value relative to two filters or energy bands.

    \subsubsection{Color.colorDef}
      \textbf{vodml-id: Color.colorDef} \newline
      \textbf{type: \hyperref[sect:ColorDef]{mango:ColorDef}} \newline
      \textbf{multiplicity: 1} \newline 
      Color definition. Can be either a difference of magnitudes or an hardness ratio.

  \subsection{ColorDef}
  \label{sect:ColorDef}
    Definition of the colour to which the Color property applies. It contains the way the colour is calculated as well as the definition of the filters used.

    \subsubsection{ColorDef.definition}
      \textbf{vodml-id: ColorDef.definition} \newline
      \textbf{type: \hyperref[sect:ColorDefinition]{mango:ColorDefinition}} \newline
      \textbf{multiplicity: 1} \newline 
      TODO : Missing description : please, update your UML model asap.

    \subsubsection{ColorDef.high}
      \textbf{vodml-id: ColorDef.high} \newline
      \textbf{type: :PhotFilter} \newline
      \textbf{multiplicity: 1} \newline 
      Reference to the \texttt{photdm:PhotFilter} corresponding the highest energy for that color.

    \subsubsection{ColorDef.low}
      \textbf{vodml-id: ColorDef.low} \newline
      \textbf{type: :PhotFilter} \newline
      \textbf{multiplicity: 1} \newline 
      Reference to the \texttt{photdm:PhotFilter} corresponding the lowest energy for that color.

  \subsection{EpochPosition}
  \label{sect:EpochPosition}
    This class is a view of \texttt{Astronomical Coordinates and Coordinate Systems} components that have been put together to form a consistent description of the position of an object moving over time. It consists of a celestial position, a proper motion, a radial velocity and a parallax. All components share the same spatial coordinate system. \begin{itemize} \item Both position and proper motion reuse the \texttt{coords:LonLatPoint} elements. \item The space coordinate system is imported from \texttt{coords:spaceSys}. \end{itemize} The error is specific to this class as it must support covariance and correlation between the components.

    \subsubsection{EpochPosition.longitude}
      \textbf{vodml-id: EpochPosition.longitude} \newline
      \textbf{type: \hyperref[sect:ivoa]{ivoa:RealQuantity}} \newline
      \textbf{multiplicity: 1} \newline 
      The longitude of the Point, as a Quantity with angular units (see \texttt{coords:LonLatPoint.lon}.

    \subsubsection{EpochPosition.latitude}
      \textbf{vodml-id: EpochPosition.latitude} \newline
      \textbf{type: \hyperref[sect:ivoa]{ivoa:RealQuantity}} \newline
      \textbf{multiplicity: 1} \newline 
      The latitude of the Point, as a Quantity with angular units (see \texttt{coords:LonLatPoint.lat}.

    \subsubsection{EpochPosition.parallax}
      \textbf{vodml-id: EpochPosition.parallax} \newline
      \textbf{type: \hyperref[sect:ivoa]{ivoa:RealQuantity}} \newline
      \textbf{multiplicity: 1} \newline 
      The measured parallax value in the coordinate system of the \texttt{EpochPosition} instance.

    \subsubsection{EpochPosition.radialVelocity}
      \textbf{vodml-id: EpochPosition.radialVelocity} \newline
      \textbf{type: \hyperref[sect:ivoa]{ivoa:RealQuantity}} \newline
      \textbf{multiplicity: 1} \newline 
      The measured Velocity value along of the radius axis (see texttt{meas:Velocity.coord}).

    \subsubsection{EpochPosition.pmLongitude}
      \textbf{vodml-id: EpochPosition.pmLongitude} \newline
      \textbf{type: \hyperref[sect:ivoa]{ivoa:RealQuantity}} \newline
      \textbf{multiplicity: 1} \newline 
      Velocity along the longitude axis in angular distance per unit time (see \texttt{meas:ProperMotion.coord}). The current version of the model only allows a representation in the Polar coordinate space.

    \subsubsection{EpochPosition.pmLatitude}
      \textbf{vodml-id: EpochPosition.pmLatitude} \newline
      \textbf{type: \hyperref[sect:ivoa]{ivoa:RealQuantity}} \newline
      \textbf{multiplicity: 1} \newline 
      Velocity along the latitude axis in angular distance per unit time (see \texttt{meas:ProperMotion.coord}). The current version of the model only allows a representation in the Polar coordinate space.

    \subsubsection{EpochPosition.epoch}
      \textbf{vodml-id: EpochPosition.epoch} \newline
      \textbf{type: coords:Epoch} \newline
      \textbf{multiplicity: 1} \newline 
      Class grouping all parameters needed to define an object position at a given epoch. The space coordinate system is common to all attributes to insure the consistance between all of the instance parameters.

    \subsubsection{EpochPosition.pmCosDeltApplied}
      \textbf{vodml-id: EpochPosition.pmCosDeltApplied} \newline
      \textbf{type: \hyperref[sect:ivoa]{ivoa:boolean}} \newline
      \textbf{multiplicity: 1} \newline 
      It is common, though not universal, practice to quote longitudinal proper motion pre-multiplied by cos(latitude) so that the magnitude of the quantity is not affected by its longitudinal position. We do not constrain the value to one form or the other in this model. Instead, this flag enables providers to convey whether or not the factor has been applied (see \texttt{meas:ProperMotion.cosLat\_applied})

    \subsubsection{EpochPosition.errors}
      \textbf{vodml-id: EpochPosition.errors} \newline
      \textbf{type: \hyperref[sect:EpochPositionErrors]{mango:EpochPositionErrors}} \newline
      \textbf{multiplicity: 0..1} \newline 
      Error coefficients of the EpochPosition attributes

    \subsubsection{EpochPosition.correlations}
      \textbf{vodml-id: EpochPosition.correlations} \newline
      \textbf{type: \hyperref[sect:EpochPositionCorrelations]{mango:EpochPositionCorrelations}} \newline
      \textbf{multiplicity: 0..1} \newline 
      Correlation coefficients between the EpochPosition attributes

    \subsubsection{EpochPosition.coordSys}
      \textbf{vodml-id: EpochPosition.coordSys} \newline
      \textbf{type: coords:SpaceSys} \newline
      \textbf{multiplicity: 0..1} \newline 
      Class grouping all parameters needed to define an object position at a given epoch. The space coordinate system is common to all attributes to insure the consistance between all of the instance parameters.

  \subsection{EpochPositionCorrelations}
  \label{sect:EpochPositionCorrelations}
    Class holder for the correlation coefficients between the EpochPosition attributes

    \subsubsection{EpochPositionCorrelations.positionPm}
      \textbf{vodml-id: EpochPositionCorrelations.positionPm} \newline
      \textbf{type: \hyperref[sect:correlation.Correlation22]{mango:correlation.Correlation22}} \newline
      \textbf{multiplicity: 0..1} \newline 
      Matrix of the correlation between the position and the proper motion

    \subsubsection{EpochPositionCorrelations.parallaxPm}
      \textbf{vodml-id: EpochPositionCorrelations.parallaxPm} \newline
      \textbf{type: \hyperref[sect:correlation.Correlation12]{mango:correlation.Correlation12}} \newline
      \textbf{multiplicity: 0..1} \newline 
      Matrix of the correlation between the parallax and the proper motion

    \subsubsection{EpochPositionCorrelations.positionParallax}
      \textbf{vodml-id: EpochPositionCorrelations.positionParallax} \newline
      \textbf{type: \hyperref[sect:correlation.Correlation21]{mango:correlation.Correlation21}} \newline
      \textbf{multiplicity: 0..1} \newline 
      Matrix of the correlation between the position and the parallax

    \subsubsection{EpochPositionCorrelations.positionPosition}
      \textbf{vodml-id: EpochPositionCorrelations.positionPosition} \newline
      \textbf{type: \hyperref[sect:correlation.Correlation22]{mango:correlation.Correlation22}} \newline
      \textbf{multiplicity: 0..1} \newline 
      Matrix of the self correlation of the position

    \subsubsection{EpochPositionCorrelations.properMotionPm}
      \textbf{vodml-id: EpochPositionCorrelations.properMotionPm} \newline
      \textbf{type: \hyperref[sect:correlation.Correlation22]{mango:correlation.Correlation22}} \newline
      \textbf{multiplicity: 0..1} \newline 
      Matrix of the self correlation of the proper motion

  \subsection{EpochPositionErrors}
  \label{sect:EpochPositionErrors}
    Class holder for the errors of the EpochPosition attributes

    \subsubsection{EpochPositionErrors.parallax}
      \textbf{vodml-id: EpochPositionErrors.parallax} \newline
      \textbf{type: \hyperref[sect:error.PropertyError1D]{mango:error.PropertyError1D}} \newline
      \textbf{multiplicity: 0..1} \newline 
      Parallax error. This error is meant to be symmetrical

    \subsubsection{EpochPositionErrors.radialVelocity}
      \textbf{vodml-id: EpochPositionErrors.radialVelocity} \newline
      \textbf{type: \hyperref[sect:error.PropertyError1D]{mango:error.PropertyError1D}} \newline
      \textbf{multiplicity: 0..1} \newline 
      Error in the radial velocity. This error is meant to be symmetrical

    \subsubsection{EpochPositionErrors.position}
      \textbf{vodml-id: EpochPositionErrors.position} \newline
      \textbf{type: \hyperref[sect:error.PropertyError2D]{mango:error.PropertyError2D}} \newline
      \textbf{multiplicity: 0..1} \newline 
      TODO : Missing description : please, update your UML model asap.

    \subsubsection{EpochPositionErrors.properMotion}
      \textbf{vodml-id: EpochPositionErrors.properMotion} \newline
      \textbf{type: \hyperref[sect:error.PropertyError2D]{mango:error.PropertyError2D}} \newline
      \textbf{multiplicity: 0..1} \newline 
      TODO : Missing description : please, update your UML model asap.

  \subsection{Label}
  \label{sect:Label}
    Free text label seen as a MANGO source property.

    \subsubsection{Label.text}
      \textbf{vodml-id: Label.text} \newline
      \textbf{type: \hyperref[sect:ivoa]{ivoa:string}} \newline
      \textbf{multiplicity: 1} \newline 
      Text of label seen as a MANGO source property.

  \subsection{MangoObject}
  \label{sect:MangoObject}
    Root class of the model. A MangoObject is something with an identifier and two docks: one for the parameters and one for the associated data.

    \subsubsection{MangoObject.identifier}
      \textbf{vodml-id: MangoObject.identifier} \newline
      \textbf{type: \hyperref[sect:ivoa]{ivoa:string}} \newline
      \textbf{multiplicity: 1} \newline 
      Unique identifier for a Source. The uniqueness of that identifier is not managed by the model. The format is free.

    \subsubsection{MangoObject.associatedDataDock}
      \textbf{vodml-id: MangoObject.associatedDataDock} \newline
      \textbf{type: \hyperref[sect:AssociatedData]{mango:AssociatedData}} \newline
      \textbf{multiplicity: 0..*} \newline 
      Collection f all data associated with the MANGO entity.

    \subsubsection{MangoObject.propertyDock}
      \textbf{vodml-id: MangoObject.propertyDock} \newline
      \textbf{type: \hyperref[sect:Property]{mango:Property}} \newline
      \textbf{multiplicity: 0..*} \newline 
      Collection of all parameters attached to the MANGO entity.

    \subsubsection{MangoObject.dataOrigin}
      \textbf{vodml-id: MangoObject.dataOrigin} \newline
      \textbf{type: \hyperref[sect:dataorigin.DataOrigin]{mango:dataorigin.DataOrigin}} \newline
      \textbf{multiplicity: 0..1} \newline 
      Description of the origin of the MANGO object. The parameters contained in this component come from the work of the DCP-IG.

  \subsection{PhotometricProperty}
  \label{sect:PhotometricProperty}
    Observed brightness of the MANGO source. The distinction between fluxes of a magnitude is made by the unit. This property should refer to a photometric calibration as defined by the \texttt{PhotDM} model.

    \subsubsection{PhotometricProperty.value}
      \textbf{vodml-id: PhotometricProperty.value} \newline
      \textbf{type: \hyperref[sect:ivoa]{ivoa:RealQuantity}} \newline
      \textbf{multiplicity: 1} \newline 
      Value of the photometric property associated with a photometric calibration as defined by the \texttt{PhotDM} model.

    \subsubsection{PhotometricProperty.photCal}
      \textbf{vodml-id: PhotometricProperty.photCal} \newline
      \textbf{type: :PhotCal} \newline
      \textbf{multiplicity: 1} \newline 
      Photometric calibration that applies to the photometric property. It must be an instance of \texttt{photdm:PhotCal}.

  \subsection{PhysicalProperty}
  \label{sect:PhysicalProperty}
    Place holder for any quantity that can be hold by measure as defined in the \texttt{Astronomical Measurements Model}.

    \subsubsection{PhysicalProperty.calibrationLevel}
      \textbf{vodml-id: PhysicalProperty.calibrationLevel} \newline
      \textbf{type: \hyperref[sect:CalibrationLevel]{mango:CalibrationLevel}} \newline
      \textbf{multiplicity: 1} \newline 
      TODO : Missing description : please, update your UML model asap.

    \subsubsection{PhysicalProperty.measure}
      \textbf{vodml-id: PhysicalProperty.measure} \newline
      \textbf{type: meas:Measure} \newline
      \textbf{multiplicity: 1} \newline 
      Instance of \texttt{Astronomical Measurements Model} that holds the Property value(s).

  \subsection{Property}
  \label{sect:Property}
    Reference to a particular measure of the MANGO entity. This class is used to specify the type of the measure as well as its role.

    \noindent \textbf{constraint} \newline
    \indent    \textbf{detail: Property.One association at the time
 }\newline


    \subsubsection{Property.semantic}
      \textbf{vodml-id: Property.semantic} \newline
      \textbf{type: \hyperref[sect:VocabularyTerm]{mango:VocabularyTerm}} \newline
      \textbf{multiplicity: 1} \newline 
      Reference to a semantic concept giving the nature of the parameter As long as the vocabulary is not set, the possible values of this attribute are given by the ParamSemantic enumeration.

    \subsubsection{Property.description}
      \textbf{vodml-id: Property.description} \newline
      \textbf{type: \hyperref[sect:ivoa]{ivoa:string}} \newline
      \textbf{multiplicity: 1} \newline 
      Free text description of the measure

    \subsubsection{Property.measure}
      \textbf{vodml-id: Property.measure} \newline
      \textbf{type: meas:Measure} \newline
      \textbf{multiplicity: 1} \newline 
      Composition link pointing to the meas:Measure instance

    \subsubsection{Property.associatedProperties}
      \textbf{vodml-id: Property.associatedProperties} \newline
      \textbf{type: \hyperref[sect:Property]{mango:Property}} \newline
      \textbf{multiplicity: 1..*} \newline 
      <Enter note text here>

  \subsection{SatusValues}
  \label{sect:SatusValues}
    TODO : Missing description : please, update your UML model asap.

    \subsubsection{SatusValues.values}
      \textbf{vodml-id: SatusValues.values} \newline
      \textbf{type: \hyperref[sect:StatusValue]{mango:StatusValue}} \newline
      \textbf{multiplicity: 1..*} \newline 
      TODO : Missing description : please, update your UML model asap.

  \subsection{Shape}
  \label{sect:Shape}
    Description of the spatial extension of the MANGO source (for e.g. dust clouds).

    \subsubsection{Shape.shape}
      \textbf{vodml-id: Shape.shape} \newline
      \textbf{type: \hyperref[sect:ivoa]{ivoa:string}} \newline
      \textbf{multiplicity: 1} \newline 
      String serialization of the spatial extension of the MANGO source.

    \subsubsection{Shape.serialization}
      \textbf{vodml-id: Shape.serialization} \newline
      \textbf{type: \hyperref[sect:ShapeSerialization]{mango:ShapeSerialization}} \newline
      \textbf{multiplicity: 1} \newline 
      Serialization mode of the spatial extension of the MANGO entity

    \subsubsection{Shape.coordSys}
      \textbf{vodml-id: Shape.coordSys} \newline
      \textbf{type: coords:SpaceSys} \newline
      \textbf{multiplicity: 0..1} \newline 
      Coordinate system the applies to the spatial extension of the MANGO source. This component is imported from the \texttt{coords} model (see \texttt{coords:SpaceSys}).

  \subsection{Status}
  \label{sect:Status}
    Property representing a status defined by a integer number that can only take on a defined number of values, each with its own description. Boolean status can be represented by \texttt{StatusValues} with 2 values e.g. 0 for False and 1 for True.

    \subsubsection{Status.status}
      \textbf{vodml-id: Status.status} \newline
      \textbf{type: \hyperref[sect:ivoa]{ivoa:integer}} \newline
      \textbf{multiplicity: 1} \newline 
      Actual value of the status.

    \subsubsection{Status.allowedValues}
      \textbf{vodml-id: Status.allowedValues} \newline
      \textbf{type: \hyperref[sect:SatusValues]{mango:SatusValues}} \newline
      \textbf{multiplicity: 0..1} \newline 
      List of the allowed values for the status. Each value has its own textual description.

  \subsection{StatusValue}
  \label{sect:StatusValue}
    List of the allowed values for the status. Each value has its own textual description.

    \subsubsection{StatusValue.value}
      \textbf{vodml-id: StatusValue.value} \newline
      \textbf{type: \hyperref[sect:ivoa]{ivoa:integer}} \newline
      \textbf{multiplicity: 1} \newline 
      TODO : Missing description : please, update your UML model asap.

    \subsubsection{StatusValue.description}
      \textbf{vodml-id: StatusValue.description} \newline
      \textbf{type: \hyperref[sect:ivoa]{ivoa:string}} \newline
      \textbf{multiplicity: 1} \newline 
      TODO : Missing description : please, update your UML model asap.

  \subsection{VocabularyTerm}
  \label{sect:VocabularyTerm}
    Term of a standardised vocabulary that applies to property.

    \subsubsection{VocabularyTerm.uri}
      \textbf{vodml-id: VocabularyTerm.uri} \newline
      \textbf{type: \hyperref[sect:ivoa]{ivoa:string}} \newline
      \textbf{multiplicity: 1} \newline 
      URI the vocabulary term.

    \subsubsection{VocabularyTerm.label}
      \textbf{vodml-id: VocabularyTerm.label} \newline
      \textbf{type: \hyperref[sect:ivoa]{ivoa:string}} \newline
      \textbf{multiplicity: 1} \newline 
      Label attached to the vocabulary term. This is necessary because the URI may not contain an explicit label. This was the case for the IUA vocabulary until the Registry WG introduced rewriting rules that fix the issue.

  \subsection{WebEndpoint}
  \label{sect:WebEndpoint}
    Associated data referenced by an URL

    \subsubsection{WebEndpoint.ContentType}
      \textbf{vodml-id: WebEndpoint.ContentType} \newline
      \textbf{type: \hyperref[sect:ivoa]{ivoa:string}} \newline
      \textbf{multiplicity: 1} \newline 
      URL mime type

    \subsubsection{WebEndpoint.url}
      \textbf{vodml-id: WebEndpoint.url} \newline
      \textbf{type: \hyperref[sect:ivoa]{ivoa:anyURI}} \newline
      \textbf{multiplicity: 1} \newline 
      Web endpoint

  \subsection{ShapeFrame}
  \label{sect:ShapeFrame}

  Possible schemes to encode a shape in a string

  \noindent \underline{Enumeration Literals}
  \vspace{-\parsep}
  \small
  \begin{itemize}
  
    \item[\textbf{STC\_S}]: \textbf{vodml-id:} ShapeFrame.STC\_S \newline
          \textbf{description:} MOC serialization
    \item[\textbf{MOC}]: \textbf{vodml-id:} ShapeFrame.MOC \newline
          \textbf{description:} STCs serialization
  \end{itemize}
  \normalsize


  \subsection{ShapeSerialization}
  \label{sect:ShapeSerialization}

  Enumeration of the supported serialization modes for the shapes

  \noindent \underline{Enumeration Literals}
  \vspace{-\parsep}
  \small
  \begin{itemize}
  
    \item[\textbf{MOC}]: \textbf{vodml-id:} ShapeSerialization.MOC \newline
          \textbf{description:} Label indicating that the shape has been serialized as a S-MOC
    \item[\textbf{STCS}]: \textbf{vodml-id:} ShapeSerialization.STCS \newline
          \textbf{description:} Label indicating that the shape has been serialized as a STCS string
    \item[\textbf{POLYGON}]: \textbf{vodml-id:} ShapeSerialization.POLYGON \newline
          \textbf{description:} Label indicating that the shape has been serialized as a polygon (cf xtypes)
  \end{itemize}
  \normalsize


  \subsection{CalibrationLevel}
  \label{sect:CalibrationLevel}

  Enumeration of different possible calibration status of the property (Obscore)

  \noindent \underline{Enumeration Literals}
  \vspace{-\parsep}
  \small
  \begin{itemize}
  
    \item[\textbf{Raw}]: \textbf{vodml-id:} CalibrationLevel.Raw \newline
          \textbf{description:} Raw instrumental data, in a proprietary or internal data provider defined format, that needs instrument specific tools to be handled (ObsCore)
    \item[\textbf{Instrumental}]: \textbf{vodml-id:} CalibrationLevel.Instrumental \newline
          \textbf{description:} Instrumental data in a standard format which could be manipulated with standard astronomical packages (ObsCore).
    \item[\textbf{Calibrated}]: \textbf{vodml-id:} CalibrationLevel.Calibrated \newline
          \textbf{description:} Science ready data with the instrument signature removed (ObsCore)
  \end{itemize}
  \normalsize


  \subsection{ColorDefinition}
  \label{sect:ColorDefinition}

  Enumeration of the different color types supported by the model.

  \noindent \underline{Enumeration Literals}
  \vspace{-\parsep}
  \small
  \begin{itemize}
  
    \item[\textbf{ColorIndex}]: \textbf{vodml-id:} ColorDefinition.ColorIndex \newline
          \textbf{description:} Difference of magnitudes: typically $M_B - M_v$
    \item[\textbf{HardnessRatio}]: \textbf{vodml-id:} ColorDefinition.HardnessRatio \newline
          \textbf{description:} Normalized ratio of fluxes: $(F_{EB2} - F_{EB1}) / (F_{EB2} + F_{EB1})$
  \end{itemize}
  \normalsize


\pagebreak
\section{Package: error }

  % INSERT FIGURE HERE
  %\begin{figure}[h]
  %\begin{center}
  %  \includegraphics[width=\textwidth]{????.png}
  %  \caption{???}\label{fig:????}
  %\end{center}
  %\end{figure}

  TODO : Missing description : please, update your UML model asap.

  \subsection{Ellipse}
  \label{sect:error.Ellipse}
    Elliptical error for 2D parameters such as the sky positions.

    \subsubsection{Ellipse.semiMajorAxis}
      \textbf{vodml-id: error.Ellipse.semiMajorAxis} \newline
      \textbf{type: \hyperref[sect:ivoa]{ivoa:real}} \newline
      \textbf{multiplicity: 1} \newline 
      Half of the ellipse major axis

    \subsubsection{Ellipse.semiMinorAxis}
      \textbf{vodml-id: error.Ellipse.semiMinorAxis} \newline
      \textbf{type: \hyperref[sect:ivoa]{ivoa:real}} \newline
      \textbf{multiplicity: 1} \newline 
      Half of the ellipse minor axis

    \subsubsection{Ellipse.angle}
      \textbf{vodml-id: error.Ellipse.angle} \newline
      \textbf{type: \hyperref[sect:ivoa]{ivoa:RealQuantity}} \newline
      \textbf{multiplicity: 1} \newline 
      Angle between the North Polar Cape (NCP) and the major axis. This angle must be positive taking into account that angles are positive from North to the East. The angle has its own unit.

  \subsection{ErrorMatrix}
  \label{sect:error.ErrorMatrix}
    Error matrix for 2D quantities. The matrix must be symetric.

    \subsubsection{ErrorMatrix.sigma1}
      \textbf{vodml-id: error.ErrorMatrix.sigma1} \newline
      \textbf{type: \hyperref[sect:ivoa]{ivoa:real}} \newline
      \textbf{multiplicity: 1} \newline 
      First diagonal element ($(x_{11})$)

    \subsubsection{ErrorMatrix.sigma2}
      \textbf{vodml-id: error.ErrorMatrix.sigma2} \newline
      \textbf{type: \hyperref[sect:ivoa]{ivoa:real}} \newline
      \textbf{multiplicity: 1} \newline 
      Second diagonal element ($(x_{11})$)

    \subsubsection{ErrorMatrix.covariance}
      \textbf{vodml-id: error.ErrorMatrix.covariance} \newline
      \textbf{type: \hyperref[sect:ivoa]{ivoa:real}} \newline
      \textbf{multiplicity: 1} \newline 
      Covariance element ($(x_{12})$ or $(x_{21})$)

    \subsubsection{ErrorMatrix.isCovariance}
      \textbf{vodml-id: error.ErrorMatrix.isCovariance} \newline
      \textbf{type: \hyperref[sect:ivoa]{ivoa:boolean}} \newline
      \textbf{multiplicity: 1} \newline 
      Boolean flag telling whether the error elements must be interpreted as covariance or matrix elements.

  \subsection{PropertyError (Abstract)}
  \label{sect:error.PropertyError}
    Root (abstract) class of the errors that can be attached to a MANGO property. The class inherits from \texttt{meas:uncertainty} in order to be usable in the context of properties based on \texttt{Measures} classes.

    \subsubsection{PropertyError.confidenceLevel}
      \textbf{vodml-id: error.PropertyError.confidenceLevel} \newline
      \textbf{type: \hyperref[sect:ivoa]{ivoa:integer}} \newline
      \textbf{multiplicity: 1} \newline 
      Confidence level of the error, expressed in $\sigma$

    \subsubsection{PropertyError.unit}
      \textbf{vodml-id: error.PropertyError.unit} \newline
      \textbf{type: \hyperref[sect:ivoa]{ivoa:Unit}} \newline
      \textbf{multiplicity: 1} \newline 
      Error unit. It must be compliant with the \texttt{VOUnit} standard. The error unit must be consistent with the unit of the property the error is attached with. This is not checked at the model level.

  \subsection{PropertyError1D}
  \label{sect:error.PropertyError1D}
    Symetrical error for 1D parameters

    \subsubsection{PropertyError1D.sigma}
      \textbf{vodml-id: error.PropertyError1D.sigma} \newline
      \textbf{type: \hyperref[sect:ivoa]{ivoa:real}} \newline
      \textbf{multiplicity: 1} \newline 
      Error amplitude.

  \subsection{PropertyError2D (Abstract)}
  \label{sect:error.PropertyError2D}
    Super (abstract) class for all errors of 2D parameters

\pagebreak
\section{Package: correlation }

  % INSERT FIGURE HERE
  %\begin{figure}[h]
  %\begin{center}
  %  \includegraphics[width=\textwidth]{????.png}
  %  \caption{???}\label{fig:????}
  %\end{center}
  %\end{figure}

  TODO : Missing description : please, update your UML model asap.

  \subsection{Correlation11}
  \label{sect:correlation.Correlation11}
    Correlation of a 1D property (A) with a 1D parameter (B): $A = a1b1 * B$

    \subsubsection{Correlation11.a1b1}
      \textbf{vodml-id: correlation.Correlation11.a1b1} \newline
      \textbf{type: \hyperref[sect:ivoa]{ivoa:real}} \newline
      \textbf{multiplicity: 1} \newline 
      Correlation coefficient giving the contribution of \texttt{B} to \texttt{A}

  \subsection{Correlation12}
  \label{sect:correlation.Correlation12}
    Correlation of a 1D property (A) with a 2D parameter (B): $A = a1b1 * B_1 + a12 * B_2$

    \subsubsection{Correlation12.a1b1}
      \textbf{vodml-id: correlation.Correlation12.a1b1} \newline
      \textbf{type: \hyperref[sect:ivoa]{ivoa:real}} \newline
      \textbf{multiplicity: 1} \newline 
      Correlation coefficient giving the contribution of the first axis of \texttt{B} to \texttt{A}

    \subsubsection{Correlation12.a1b2}
      \textbf{vodml-id: correlation.Correlation12.a1b2} \newline
      \textbf{type: \hyperref[sect:ivoa]{ivoa:real}} \newline
      \textbf{multiplicity: 1} \newline 
      Correlation coefficient giving the contribution of the second axis of \texttt{B} to \texttt{A}

  \subsection{Correlation21}
  \label{sect:correlation.Correlation21}
    Correlation of a 2D property (A) with a 1D parameter (B): $A_1 = a11 * B$ $A_2 = a21 * B$

    \subsubsection{Correlation21.a2b1}
      \textbf{vodml-id: correlation.Correlation21.a2b1} \newline
      \textbf{type: \hyperref[sect:ivoa]{ivoa:real}} \newline
      \textbf{multiplicity: 1} \newline 
      Correlation coefficient giving the contribution of \texttt{B} to the second axis of \texttt{A}

    \subsubsection{Correlation21.a1b1}
      \textbf{vodml-id: correlation.Correlation21.a1b1} \newline
      \textbf{type: \hyperref[sect:ivoa]{ivoa:real}} \newline
      \textbf{multiplicity: 1} \newline 
      Correlation coefficient giving the contribution of \texttt{B} to the first axis of \texttt{A}

  \subsection{Correlation22}
  \label{sect:correlation.Correlation22}
    Correlation of a 2D property (A) with a 2D parameter (B): $A_1 = a11 * B_1 + a12 * B_2 $ $A_2 = a21 * B_1 + a22 * B_2 $

    \subsubsection{Correlation22.a2b1}
      \textbf{vodml-id: correlation.Correlation22.a2b1} \newline
      \textbf{type: \hyperref[sect:ivoa]{ivoa:real}} \newline
      \textbf{multiplicity: 1} \newline 
      Correlation coefficient giving the contribution of the second axis of \texttt{B} to the first axis of \texttt{A}.

    \subsubsection{Correlation22.a2b2}
      \textbf{vodml-id: correlation.Correlation22.a2b2} \newline
      \textbf{type: \hyperref[sect:ivoa]{ivoa:real}} \newline
      \textbf{multiplicity: 1} \newline 
      Correlation coefficient giving the contribution of the second axis of \texttt{B} to the second axis of \texttt{A}.

    \subsubsection{Correlation22.a1b1}
      \textbf{vodml-id: correlation.Correlation22.a1b1} \newline
      \textbf{type: \hyperref[sect:ivoa]{ivoa:real}} \newline
      \textbf{multiplicity: 1} \newline 
      Correlation coefficient giving the contribution of the first axis of \texttt{B} to the first axis of \texttt{A}.

    \subsubsection{Correlation22.a1b2}
      \textbf{vodml-id: correlation.Correlation22.a1b2} \newline
      \textbf{type: \hyperref[sect:ivoa]{ivoa:real}} \newline
      \textbf{multiplicity: 1} \newline 
      Correlation coefficient giving the contribution of the second axis of \texttt{B} to the first axis of \texttt{A}.

  \subsection{QuantityCorrelation (Abstract)}
  \label{sect:correlation.QuantityCorrelation}
    Abstract type for the correlation parameters of 2 quantities

    \subsubsection{QuantityCorrelation.isCovariance}
      \textbf{vodml-id: correlation.QuantityCorrelation.isCovariance} \newline
      \textbf{type: \hyperref[sect:ivoa]{ivoa:boolean}} \newline
      \textbf{multiplicity: 0..1} \newline 
      Boolean telling whether the correlations must be interpreted as covariance or as correlation coefficients.

\pagebreak
\section{Package: dataorigin }

  % INSERT FIGURE HERE
  %\begin{figure}[h]
  %\begin{center}
  %  \includegraphics[width=\textwidth]{????.png}
  %  \caption{???}\label{fig:????}
  %\end{center}
  %\end{figure}

  TODO : Missing description : please, update your UML model asap.

  \subsection{Article}
  \label{sect:dataorigin.Article}
    Reference article for the MANGO entity

    \subsubsection{Article.editor}
      \textbf{vodml-id: dataorigin.Article.editor} \newline
      \textbf{type: \hyperref[sect:ivoa]{ivoa:string}} \newline
      \textbf{multiplicity: 1} \newline 
      Article editor name

    \subsubsection{Article.article}
      \textbf{vodml-id: dataorigin.Article.article} \newline
      \textbf{type: \hyperref[sect:ivoa]{ivoa:string}} \newline
      \textbf{multiplicity: 1} \newline 
      Bibcode or DOI of the reference article

  \subsection{DataOrigin}
  \label{sect:dataorigin.DataOrigin}
    Class representing the origin of the data following the DCP note (TBD)

    \subsubsection{DataOrigin.citation}
      \textbf{vodml-id: dataorigin.DataOrigin.citation} \newline
      \textbf{type: \hyperref[sect:ivoa]{ivoa:string}} \newline
      \textbf{multiplicity: 1} \newline 
      Dataset identifier that can be used for citation (e.g. DOI)

    \subsubsection{DataOrigin.reference\_url}
      \textbf{vodml-id: dataorigin.DataOrigin.reference\_url} \newline
      \textbf{type: \hyperref[sect:ivoa]{ivoa:string}} \newline
      \textbf{multiplicity: 1} \newline 
      Dataset landing page

    \subsubsection{DataOrigin.resource\_version}
      \textbf{vodml-id: dataorigin.DataOrigin.resource\_version} \newline
      \textbf{type: \hyperref[sect:ivoa]{ivoa:string}} \newline
      \textbf{multiplicity: 1} \newline 
      Dataset version

    \subsubsection{DataOrigin.creator}
      \textbf{vodml-id: dataorigin.DataOrigin.creator} \newline
      \textbf{type: \hyperref[sect:ivoa]{ivoa:string}} \newline
      \textbf{multiplicity: 1} \newline 
      Person(s) mainly involved in the creation of the resource, generally the author

    \subsubsection{DataOrigin.cites}
      \textbf{vodml-id: dataorigin.DataOrigin.cites} \newline
      \textbf{type: \hyperref[sect:ivoa]{ivoa:string}} \newline
      \textbf{multiplicity: 1} \newline 
      Identifier (IVOID, DOI or Bibcode) of a second Resource using relation of type \texttt{cites} (\url{https://www.ivoa.net/rdf/voresource/relationship\_type/})

    \subsubsection{DataOrigin.is\_derived\_from}
      \textbf{vodml-id: dataorigin.DataOrigin.is\_derived\_from} \newline
      \textbf{type: \hyperref[sect:ivoa]{ivoa:string}} \newline
      \textbf{multiplicity: 1} \newline 
      Identifier (IVOID, DOI or Bibcode) of a second resource using relation of type \texttt{is\_derived\_from} (\url{https://www.ivoa.net/rdf/voresource/relationship\_type/})

    \subsubsection{DataOrigin.original\_date}
      \textbf{vodml-id: dataorigin.DataOrigin.original\_date} \newline
      \textbf{type: \hyperref[sect:ivoa]{ivoa:datetime}} \newline
      \textbf{multiplicity: 1} \newline 
      Date of the original resource from which the MANGO instance is derived

    \subsubsection{DataOrigin.query}
      \textbf{vodml-id: dataorigin.DataOrigin.query} \newline
      \textbf{type: \hyperref[sect:dataorigin.QueryOrigin]{mango:dataorigin.QueryOrigin}} \newline
      \textbf{multiplicity: 0..1} \newline 
      TODO : Missing description : please, update your UML model asap.

    \subsubsection{DataOrigin.rights}
      \textbf{vodml-id: dataorigin.DataOrigin.rights} \newline
      \textbf{type: \hyperref[sect:dataorigin.License]{mango:dataorigin.License}} \newline
      \textbf{multiplicity: 0..1} \newline 
      TODO : Missing description : please, update your UML model asap.

    \subsubsection{DataOrigin.article}
      \textbf{vodml-id: dataorigin.DataOrigin.article} \newline
      \textbf{type: \hyperref[sect:dataorigin.Article]{mango:dataorigin.Article}} \newline
      \textbf{multiplicity: 0..1} \newline 
      TODO : Missing description : please, update your UML model asap.

  \subsection{License}
  \label{sect:dataorigin.License}
    Place holder for the license covering the MANGO instance

    \subsubsection{License.rights\_uri}
      \textbf{vodml-id: dataorigin.License.rights\_uri} \newline
      \textbf{type: \hyperref[sect:ivoa]{ivoa:string}} \newline
      \textbf{multiplicity: 1} \newline 
      Licence URI following the regostry practice. This should come from SPDX \url{https://spdx.org/licenses}. Creatives Commons (\url{https://creativecommons.org}) are also accepted.

    \subsubsection{License.rights}
      \textbf{vodml-id: dataorigin.License.rights} \newline
      \textbf{type: \hyperref[sect:ivoa]{ivoa:string}} \newline
      \textbf{multiplicity: 1} \newline 
      License or Copyright text

  \subsection{QueryOrigin}
  \label{sect:dataorigin.QueryOrigin}
    Description of the query the MANGO instance results from.

    \subsubsection{QueryOrigin.ivoid}
      \textbf{vodml-id: dataorigin.QueryOrigin.ivoid} \newline
      \textbf{type: \hyperref[sect:ivoa]{ivoa:string}} \newline
      \textbf{multiplicity: 1} \newline 
      IVOID of the underlying data collection

    \subsubsection{QueryOrigin.publisher}
      \textbf{vodml-id: dataorigin.QueryOrigin.publisher} \newline
      \textbf{type: \hyperref[sect:ivoa]{ivoa:string}} \newline
      \textbf{multiplicity: 1} \newline 
      Data center that produced the MANGO instance

    \subsubsection{QueryOrigin.server\_software}
      \textbf{vodml-id: dataorigin.QueryOrigin.server\_software} \newline
      \textbf{type: \hyperref[sect:ivoa]{ivoa:string}} \newline
      \textbf{multiplicity: 1} \newline 
      Version of the software the produced the MANO source instance. It is encouraged to follow \url{https://ivoa.net/documents/Notes/softid/index.html}.

    \subsubsection{QueryOrigin.service\_protocol}
      \textbf{vodml-id: dataorigin.QueryOrigin.service\_protocol} \newline
      \textbf{type: \hyperref[sect:ivoa]{ivoa:string}} \newline
      \textbf{multiplicity: 1} \newline 
      IVOID of the protocol through which the data was retrieved

    \subsubsection{QueryOrigin.request}
      \textbf{vodml-id: dataorigin.QueryOrigin.request} \newline
      \textbf{type: \hyperref[sect:ivoa]{ivoa:string}} \newline
      \textbf{multiplicity: 1} \newline 
      Full request URL including a query string. For the simple protocols,put the url-encoded form of the query parameters. For TAP queries, use the /sync UWS URL. The format is free for others request types.

    \subsubsection{QueryOrigin.query}
      \textbf{vodml-id: dataorigin.QueryOrigin.query} \newline
      \textbf{type: \hyperref[sect:ivoa]{ivoa:string}} \newline
      \textbf{multiplicity: 1} \newline 
      Input query in a formal langage such as ADQL.equest types

    \subsubsection{QueryOrigin.request\_date}
      \textbf{vodml-id: dataorigin.QueryOrigin.request\_date} \newline
      \textbf{type: \hyperref[sect:ivoa]{ivoa:datetime}} \newline
      \textbf{multiplicity: 1} \newline 
      Query execution date

    \subsubsection{QueryOrigin.contact}
      \textbf{vodml-id: dataorigin.QueryOrigin.contact} \newline
      \textbf{type: \hyperref[sect:ivoa]{ivoa:string}} \newline
      \textbf{multiplicity: 1} \newline 
      Email or URL to contact the publisher