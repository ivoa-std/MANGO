This non-normative section describes potential strategies for
storing and discovering MANGO instances within TAP services.
We assume that the TAP service hosts catalogs whose data can be mapped to MANGO;
such catalogs are designated as \emph{MANGO catalogues}.

\subsection{Storing MANGO Catalogs in TAP}

There are many ways to implement the mapping of TAP query responses on MANGO.
This topic has been discussed in ADASS 2022 \citep{2024ASPC..535..259M}.

In this section we assume that this mapping is done with MIVOT, which means that the server has to create an 
XML block above the data that contains a model view bound with that queried data.

The way to proceed is a design choice that can also be influenced by the TAP 
framework currently in use (VOLLT, DaCHS, OpenCADC ...). 
However, we can identify some features that need to be implemented in some way, 
either by the developers or by the curators, to get the job done:

\begin{enumerate}
    \item A storage system that associates table columns with the attributes 
          of the MANGO properties to which they can be mapped. 
          This can be done by adding specific columns to the \texttt{TAP\_SCHEMA.columns} 
          table or by using specific resources such as configuration files stored out of the TAP\_SCHEMA.
          This binding must also support metadata that are possibly not in the TAP\_SCHEMA:
          \begin{itemize}
              \item Reference to the coordinate frame or the photometric systems
              \item Missing metadata such as the semantics, the confidence levels or the data origin
              \item Error classes (ellipse, matrix...)
              \item Possible links binding MANGO properties together.
          \end{itemize}
     \item A module capable of providing the system with XML snippets corresponding to the MANGO 
           properties to be stacked in the query response. These snippets can be 
           dynamically generated from the VO-DML files or stored locally in individual files.
     \item A connector to the ADQL parser that is capable of returning the table columns being queried.
           This module must be able to distinguish between table columns and computed columns,
           it must also be able to deal with column aliases.
\end{enumerate}

Once these are available, the mapper module can use 1) and 3) to get the properties to be included 
in the response and 2) to build the mapping block. 

\subsection{ \emph{MANGOCore} Table}

MANGO could also be used as a convenient way to find out what properties are available for the tables being served,
or which tables come with the desired properties (e.g. give me all tables with both proper motion and G magnitude).

This could be achieved with a \emph{MANGOCore} table located in the \emph{ivoa} schema. 
This table would have one rows per exposed table and one (or more) columns per exposed property.
The fact that the MANGO properties are exposed as open collections complicated the definition 
of the \emph{MANGOCore} table, and at some point it will be necessary to choose which property 
to expose (or not) with this table.
This \emph{MANGOCore} implementation topic is beyond the scope of the present document but the concept
is another valuable use of MANGO.  



