Right now, the meta-data provided within the VOTable allow clients such Aladin or Topcat to run most 
of the functionalities expected by the user, either for data analysis of plotting.
This information is often guess from UCDs, UTypes or columns name. It can also be given by the user.
Clients have no expectations of working with full model instances but in some cases models 
can help to know how quantities in an input table relate to each other.
In most cases this is for visualisation, e.g.:
\begin{itemize}
    \item what is the sky position for this row
    (what columns contain latitude and longitude, and what sky system are they in)

     \item what +/-ERR error bars should I plot for these points
    (what column is a simple error for column A)

    \item what error ellipses should I plot for these sky positions
    (what columns provide ra\_error, dec\_error, ra\_dec\_corr,
     or how can I derive those from columns that do exist)

    \item where do I get the grid information for a column containing
    a vector of samples so I can label the X axis of a spectrogram
    (what column or parameter contains an axis vector matching
     the sample vectors)

    \item does this table contain sky positions, or HEALPix tiles, or both?
    What's the best way to represent it on the sky?

    \item What is the meaning of such URL found out in a table?s
\end{itemize}

But there are some other places too:
\begin{itemize}
    \item how do I propagate this sky position to a future epoch
    (what columns contain pmra, pmdec, and maybe all the
     associated errors and correlation coefficients)

    \item what is the error ellipse/oid to use for a sky/Cartesian crossmatch
    (what columns provide the relevant errors and, if available,
     correlations)
\end{itemize}

This usage shows that MANGO must be designed in a way that individual measurements or quantities can be easily be identified as such and manipulated independently of the whole instance.
