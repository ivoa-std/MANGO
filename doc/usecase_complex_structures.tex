The ViaLactea Knowledge Base (VLKB, see \cite{2016SPIE.9913E..0HM}) is a set of data
resources and services built up to study the star formation regions and
processes in the Milky Way. Besides 2-D images and 3-D radial velocity
cubes, the VLKB exposes a bunch of source catalogues. These catalogues
can be categorised as:

\begin{itemize}
	\item compact sources catalogues: where each source can be simply
described by position and (usually) ellipsoidal extension plus some
photometric flux(es);
	\item a band merged catalogue: where each source record combines the
single band source of the previous type into a
multi-center/multi-positional record;
	\item diffuse objects of two types, bubbles and filaments, where the
positional description cannot be simplified by a point in the sky and
some geometric error/shape value.
\end{itemize}

The first type of catalogue presents only a small issue, related to the
ellipsoidal description of the extension of the single records/sources.
For these catalogues the ellipsoidal (or circular) counterpart of the position is not
to be considered as a positional uncertainty, but actually some level of
confidence on the actual extension of the source at the observed
frequency band, thus, besides a position on the celestial sphere, a minor axis,
a major axis and a positional angle are needed to describe the source.

The band merged catalogue requires an extra step, that is the
aggregagtion of multiple single band sources (potentially degenerate
depending on the observational waveband) to be able to provide a sort
of SED of the aggregated source.

A different type of challenge is presented by filaments and bubbles
where (not approximating bubbles to simple circles) the source
morphology requires or a complex geometrical shape, like single or
multiple polygons, or a description based on tessellation (where the
order of the tessels should fit the resolution/uncertainty of the source
borders). Moreover, while bubbles can be fully described by a centroid
and a single complex (multi-)polygon or set of tessels, filaments are
themselves complex objects such that their description includes
so-called branches, spines and nodes (areas, broken-lines, points in
spherical geometry).

Thus, a model that supports description of such catalogues will need a
way to describe sources with:
\begin{itemize}
	\item non-point-like positions;
	\item extended complex area, possibly as multiple detached areas;
	\item aggregation of sub-parts (that can be heterogeneous).
\end{itemize}

The VLKB tried\footnote{if you want to have a look at the plain
content, there's a TAP service at
http://vlkb.dev.neanias.eu:8080/vlkb/tap (http://saada.unistra.fr/taphandle?url=http\%3A//vlkb.dev.neanias.eu\%3A8080/vlkb/tap/) that can be accessed, even if
not complete with descriptions nor registered} to describe, both by
(custom) contour polygons and MOCs,
the bubbles and filaments having the main use case of cross-matching or
distance measuring diffuse objects w.r.t. compact sources. However, work
progressend slowly and anyway the proper model annotations were never
investigated properly.

MANGO could be useful to serve nicely annotated and usable records for
sources as the ones described above.