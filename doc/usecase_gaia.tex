Gaia mission is producing the largest and more precise 3D map of our galaxy.

Gaia core solution is able to solve the astrometric solution of more than 1
billion sources by complex models and algorithms \citep{2012A&A...538A..78L}.
Using a minimisation problem approach, different detections identified on
different scans can be associated to the same astronomical source. Some of the
properties would be direct measurements on single scans (e.g. positions or
magnitudes). Also other properties like radial velocity (measured in redshift
units) are also obtained at integration time of the scans.

Once detections on different scans are associated to a single astronomical
source, these direct measurements can be combined to generate derived
measurements. Trajectories on the sky for galactic objects are seen as
spirals including the combination of the proper motion of the object
(derived by the main vector of movement of the source) and the parallax
(derived by the radius of the apparent spiral produced by the different
angles of the Gaia observations at different periods).

From these properties, others could be also derived like, e.g. the distance,
although the exact transformation from parallax to distance requires the use of
accurate calculations and calibrations so, in general, only direct astrometric
properties, like parallax, are usually provided into the catalogues.

Finally, other properties will be also obtained by the cross-identification of
detections as a single astronomical source. For example, time series could be
combined as the result of measurements of magnitudes from different scans
detections.

Although it is not its main scientific target and apart from stellar objects,
other extra-galactic sources could be also studied with Gaia. For example, QSOs
are observed by Gaia as point-like sources with zero proper motion. For these
kind of sources, typical extra-galactic properties like, redshift, could be also
provided.

A non-exhaustive list of properties required for Gaia use cases would be composed
of:

\begin{itemize}
    \item identifier
    \item sky reference position
    \item proper motion
    \item parallax and distance

    \item source extension
    \item radial velocity
    \item redshift
    \item photometry
    \item date of detection
    \item correlation
    \item multiple detection
\end{itemize}