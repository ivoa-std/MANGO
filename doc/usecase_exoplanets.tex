Annotation of (exo-)planetary records in catalogues requires some
specific metadata or model.

This was initially made clear when trying to
allow discoverability of exoplanets time series
(\cite{2020ASPC..522..703M}) where (exercising the
ObsCore model with specific use cases) information like stellar host
characteristics together with stellar system details turned out to be required for a
proper description of the catalogue records.

The use cases identified,
besides some refinement needed in the ObsCore time axis annotation, the
following metadata:
\begin{itemize}
	\item the degree of confidence in the detection: exoplanets candidates
w.r.t. confirmed ones, plus last update of the record content ;
	\item the method used in the discovery (since it affects the available
stellar system description parameters);
	\item a set of stellar host characteristics (besides sky coordinates):
activity, mass, type,
metallicity, age, some systemic values, like the global RV (radial
velocity) of the system, and so on;
	\item (exo-)planet parameters, like mass, orbital period, orbit's
eccentricity, RV semi-amplitude, time at periastron (for RV detections)
or central transit time (for transit method), longitude of periastron,
and so on.
\end{itemize}

The scenario gained some further complexity, but also better stated the
idea of modelling datasets and catalogue records in the (exo-)planetary
systems subdomain, incorporating requirements from exoplanets atmosphere
simulations and (first efforts) observations.

Specific metadata additions were not specified exactly, but a draft model
to describe stellar systems was developed, specifically trying to solve
the issue in adding metadata elements to describing orbiting celestial
objects (\cite{2019ASPC..523..597M}).

The model identified the main concepts and classes as:
\begin{itemize}
	\item Celestial Object, typed/subclassed as: Star, Binary Star,
Planet, Satellite, Brown Dwarf, Trojan, and so on;
	\item Orbit, as the class keeping the information needed to describe
the orbit of a couple (or more) Celestial Objects.
\end{itemize}

Specific (exo-)planets metadata included: atmosphere with molecular
composition, bulk details (mass, radius, \ldots).

A catalogue service that could make use of model metadata is the
Exo-MerCat (\cite{2020A&C....3100370A}) catalogue (or the other available exoplanets'
catalogues), already available in a VO-aware
solution\footnote{ivo://ia2.inaf.it/catalogues/exomercat}, but using only
simple UCD description of the exposed information.

\TODO{mention the involved projects : examples ? GAPS, TESS? }
\TODO{For GAPS I can add specifics later, if implementation can
(re)start}