The basic cross-match of two astronomical tables consists in associating pairs of sources -- one 
from each table -- fulfilling a given angular distance based criterion.
In relational algebra terms, it is a theta-join on a distance predicate.

More generally, a cross-match is the association of sources from different tables given their 
proximity in an astrometric (but also possibly photometric, statistical, ...) parameter space \citep{2017A&A...597A..89P} .

If proper motions (plus parallax and radial velocities) are available, the cross-match tool 
may propagate the positions of each table to a common epoch.
It may also take into account positional uncertainties to reject the statistically unlikely associations.

In the latter case (cross-match between two tables taking into account positional errors),
the tool needs to be able to retrieve the errors associated to the each position in each table.

UCDs may help in identifying the errors associated to a positional columns as shown in table \ref{tbl::pos-errors}.


\begin{table}[!htbp]
\small
\centering
\begin{tabulary}{\linewidth}{|L|C|J|}
       \hline
           \textbf{Error type  \newline Parameters} &
           \textbf{UCD} &
           \textbf{Description} \\
       \hline
           \multicolumn{3}{c}{\textbf{Circular error}} \\
       \hline
           \texttt{epos}&

           \texttt{stat.error;pos.eq}    &
           See "possible ambiguity for circular errors" \\
        \hline
           \multicolumn{3}{c}{\textbf{Uncorrelated errors}} \\
        \hline
           \texttt{eRA}     &
           \texttt{stat.error;pos.eq.ra}    &
            Error on `RA cos(Dec)` \\
         \hline
            \texttt{eDec}  &
           \texttt{stat.error;pos.eq.dec}    &
            Error on `Dec`\\
         \hline
            \multicolumn{3}{c}{\textbf{Correlated errors}} \\
        \hline
           \texttt{eRa}  &
           \texttt{stat.error;pos.eq.ra}    &
           Error on `RA cos(Dec)` \\
        \hline
           \texttt{eDec}&
           \texttt{stat.error;pos.eq.dec}   &
           Error on `Dec`     \\
        \hline
            \texttt{corRADec}&
           \texttt{stat.covariance; pos.eq.ra; pos.eq.dec }    &
            Correlation factor\\
        \hline
             \multicolumn{3}{c}{\textbf{Oriented Ellipse}} \\
        \hline
           \texttt{a}   &
           \texttt{phys.angSize.smajAxis; pos.errorEllipse}    &
           Error ellipse semi-major axis \\
        \hline
            \texttt{b} &
           \texttt{phys.angSize.sminAxis; pos.errorEllipse }    &
            Error ellipse semi-minor axis \\
        \hline
            \texttt{theta}  &
           \texttt{pos.posAng; pos.errorEllipse}    &
           Error ellipse position angle \\
       \hline
\end{tabulary}
\caption{Table of the different possible representations of positional errors that can be found in astronomical catalogues}
\label{tbl::pos-errors}
\end{table}

But this is not sufficient to table with more complex cases based on multi-parameter cases:

\begin{itemize}
	\item Catalogues like AllWISE provides a co-sigma instead of the correlation factor of the covariance matrix.
	Co-sigma is the sign of the correlation factor time the square root of the covariance.
	\item Table fields UCDs  may be too loose: for example  \texttt{stat.error;pos.eq} is often used in place of \texttt{phys.angSize.smajAxis;pos.errorEllipse} or \texttt{phys.angSize.sminAxis;pos.errorEllipse}.
	\item The location  of the column to be used for the Xmatch can be ambiguous. For instance, if several pairs of position are provided in a table, there is currently
	no way to associate unambiguously uncertainties with the (right) pair of coordinates.
	\item  Possible ambiguity for circular errors. When the provided uncertainty is the parameter of a circular error, it may be:
	\begin{itemize}
		\item either the 1 dimensional component on each axis of a symmetric 2-dimensional Gaussian distribution: $\sigma=\sigma_{\alpha\cos\delta}=\sigma_\delta$;
		\item or the parameter of the radial error distribution (i.e. of the Rayleigh distribution): $\sigma=\sqrt{\sigma_{\alpha\cos\delta}^2 + \sigma_\delta^2}$
                 with $\sigma_{\alpha\cos\delta} = \sigma_\delta$
         \end{itemize}

	\item   Possible ambiguity on the confidence level
	The provided error is usually the 1sigma error.
	It (theoretically) means that the "true" position has:
	\begin{itemize}
		\item either 68\% chances to be at a distance lower than the radial error from the position's mean value.
		\item or 39\% chances to be inside the error ellipse (or circle) around the position's mean value.
	\end{itemize}
	But depending on the catalogue, the provided error parameters can correspond to different confidence levels.

\end{itemize}

