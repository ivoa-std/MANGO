The main purpose of MANGO is to add an upper description level to the tabular data of query responses.
MANGO is not designed to replace the meta-data already present in query responses, but on the contrary, 
to provide a model-aware layer with structured classes to interpret them and exploit them in client applications. 

Uses-cases have been collected since 2019 from representatives of various astronomical 
missions, archive designers and tools developers.
The call for contribution was totally open. This gave a good picture of the needs but we do not pretend 
that everything will be supported by this first version.
All the use-cases summarized below are detailed in appendix.

\subsubsection{GAIA}
The GAIA mission is producing the largest and more precise 3D map of our galaxy.
The GAIA Astrometric Core Solution is able to provide the astrometry of more than 1
billion sources by complex models and algorithms \citep{2012A&A...538A..78L}.
Using a minimization problem approach, different detections identified on
different scans can be associated to the appropriate astronomical source. Some of the
properties would be direct measurements on single scans (e.g. positions or
magnitudes). Other properties like radial velocity (measured in redshift
units) are also obtained at integration time of the scans.

A non-exhaustive list of properties required for GAIA use cases would be composed
of:

\begin{itemize}[noitemsep,topsep=0pt,parsep=0pt,partopsep=0pt]
    \item detection identifier
    \item sky reference position
    \item proper motion
    \item parallax and distance

    \item source extension
    \item radial velocity
    \item redshift
    \item photometry
    \item date of detection
    \item correlation
    \item multiple detection
\end{itemize}

\subsubsection{Euclid}
The Euclid telescope has been designed to unveil some of the questions about the
dark Universe, including dark matter and dark energy, what would include, f.i
quite accurate measurements of the expansion of the Universe.

Euclid will mainly observe extragalactic objects providing, f.i information
about the shapes of galaxies, gravitational lensing, baryon acoustic oscillations
and distances to galaxies using spectroscopic data.

For this mission, and apart from the common metadata provided for extra galactic
sources into astronomical catalogues, a good support for object taxonomy and
shapes of objects will be required. As known due to general relativity effects,
shapes of far galaxies could be deformed due to gravitational lensing effects,
producing convergence (visual displacements on the position) and rear (deformation
of the shape) effects. All these metadata should be ready for annotations and,
also, correlated to theoretical or real metadata in other datasets.

Finally, crossmatch information with other catalogues will be of crucial interest
as data from other satellites and, more importantly, from ground based
observatories will be combined with Euclid data to produce consistent scientific
datasets.

Typical features for objects entail: 

\begin{itemize}[noitemsep,topsep=0pt,parsep=0pt,partopsep=0pt]
    \item identifier
    \item sky position
    \item correlation with other catalogues
    \item photometry (ground + satellite )
    \item morphology class
    \item redshift
    \item photometric redshift
\end{itemize}

\subsubsection{Exoplanets}
Annotation of (exo-)planetary records in catalogues requires some
specific metadata or model.

The use cases identified requires the following metadata:
\begin{itemize}[noitemsep,topsep=0pt,parsep=0pt,partopsep=0pt]
	\item the degree of confidence in the detection: exoplanets candidates
w.r.t. confirmed ones, plus last update of the record content ;
	\item the method used in the discovery (since it affects the available
stellar system description parameters);
	\item a set of stellar host characteristics (besides sky coordinates):
activity, mass, type,
metallicity, age, some systemic values, like the global RV (radial
velocity) of the system, and so on;
	\item (exo-)planet parameters, like mass, orbital period, orbit's
eccentricity, RV semi-amplitude, time at periastron (for RV detections)
or central transit time (for transit method), longitude of periastron,
and so on.
\end{itemize}
 
 
\subsubsection{Morphologically Complex Structures}
The ViaLactea Knowledge Base (VLKB, see \cite{2016SPIE.9913E..0HM}) is a set of data
resources and services built up to study the star formation regions and
processes in the Milky Way. Besides 2-D images and 3-D radial velocity
cubes, the VLKB exposes a bunch of source catalogues.
A model that supports description of such catalogues will need a
way to describe sources with:
\begin{itemize}[noitemsep,topsep=0pt,parsep=0pt,partopsep=0pt]
	\item non-point-like positions;
	\item extended complex area, possibly as multiple detached areas;
	\item aggregation of sub-parts (that can be heterogeneous).
\end{itemize}

\subsubsection{X-ray Observatory Archives}

The requirements for both Chandra (get more in appendix \ref{sec:chandra})
and XMM-Newton \footnote{https://www.cosmos.esa.int/web/xmm-newton} science cases
are combined in this use case.
These 2 X-ray observatories have many common features that could take advantage of sharing the same model:

\begin{itemize}[noitemsep,topsep=0pt,parsep=0pt,partopsep=0pt]
    \item Both work as photon counters with a good time resolution.
          The result is that physical quantities remain tied to the instrument response.
          Therefore, the metadata must refer to instrumental parameters that are needed to
          understand the data well.
    \item Both observatories work in pointed mode and provide the community with sets of products per observation.
    \item Observation-level data are periodically merged into catalog of detections, 
          which is a very important scientific product,
          but individual observations are equally important and are used directly for analysis. 
     \item Detection catalogs are merged into source catalogs, and it is important to be able to
           associate sources with their detections.
     \item Equally important, given the more than 2 decades that both spacecraft are flying,
           is the ability to correlate catalog data with time.
     \item X-ray data reveal quantities that are usually not well supported by the VO: 
           \begin{itemize}
               \item energy bands
               \item hardness ratio
               \item Flags that are very important for understanding the source detections.
               \item Complex errors (asymmetric, ellipse)
               \item model-based data (flux, spectra) 
           \end{itemize}
    \item X-ray data are often analyzed in conjunction with data from other domains, 
          This is made easier if they all have the same way of describing the quantities of interest.
\end{itemize}

% Ian E. Mail (17/10)========================

% The CSC does provide independent lower and upper confidence limits for each measurement as part of the data tables. 
% They are separate columns for us (eg, we have measurement, measurement_lolim, and measurement_hilim as 3 columns) 
% but I wonder if these can be handled as a single concept in MANGO?
% For positions on the sky we similarly use a position error ellipse with defined semi-axes and orientation.

% The one other thing I think about is that we often have many very similar measurements in the CSC.  
% For example, for aperture photometry, we have multiple energy bands,
% and in each energy band we measure flux in the aperture in multiple ways
% (eg, photon flux, energy flux, model energy flux (based on several canonical
% spectral models such as absorbed power-law, absorbed black-body, …),
% spectral fit energy flux (based on several spectral models where the parameters are 
% fitted to the data - requires more counts to get robust fits). 
% And we may do this for multiple configurations of individual observations of a source 
% (eg, a straight average - usually for comparison with other catalogs, or a set based on a multi-band Bayesian 
% Blocks analysis -
% so we’re grouping observations in which the source has constant flux in each of the energy bands).  
% How we represent these many different types of very similar measurements in a way that is scientifically useful
% and searchable is complex.  Can this even be done usefully using UCDs?

% ============================================

\subsubsection{Vizier catalog archive}
VizieR provides science ready catalogs coming from space agencies or articles from the astronomical journals, covering number of different science cases.
Published data encompass a very large set of measures (position, photometry, redshift, source type, etc.)
depending on their origin.
They can result from  observations, simulations, models or catalog compilations.
Individual Vizier tables can contain data all related to one source (e.g. time series of positions or magnitudes) or to a set of sources (one row per source) or a mix of both.

The MANGO model must be able to provide a standard representation of most of the metadata contained 
in Vizier query responses, either native or computed by the CDS, and organized either as 
simple quantities or as associated complex data.
MANGO is not meant to replace the current management of the ViZier metadata, but rather 
to make those understandable/interoperable for a wide panel of VO-compliant clients.

\subsubsection{Client (on Mark Taylor behalf)}
Right now, the meta-data provided within the VOTable allow client software such Aladin or Topcat to run most 
of the functionalities expected by the user, either for data analysis or plotting.
This information is often inferred from UCDs, UTypes or column names. It can also be given by the user.
Client applications do not require working with full model instances but in some cases models 
can make it explicit how quantities in an input table relate to each other.

Most cases are oriented towards interpretation of columns for visualization, e.g.:
\begin{itemize}[noitemsep,topsep=0pt,parsep=0pt,partopsep=0pt]
    \item what is the sky position for this row
    (what columns contain latitude and longitude, and what sky system are they in)

     \item what +/-ERR error bars should I plot for these points
    (what column is a simple error for column A)

    \item what error ellipses should I plot for these sky positions
    (what columns provide ra\_error, dec\_error, ra\_dec\_corr,
     or how can I derive those from columns that do exist)

    \item where do I get the grid information for a column containing
    a vector of samples so I can label the X axis of a spectrogram
    (what column or parameter contains an axis vector matching
     the sample vectors)

    \item does this table contain sky positions, or HEALPix tiles, or both?
    What's the best way to represent it on the sky?

    \item what is the meaning of such URL found out in a table?s
\end{itemize}

But there are some other cases like:
\begin{itemize}[noitemsep,topsep=0pt,parsep=0pt,partopsep=0pt]
    \item how do I propagate this sky position to a future epoch
    (what columns contain pmra, pmdec, and maybe all the
     associated errors and correlation coefficients)

    \item what is the error ellipse/oid to use for a sky/Cartesian crossmatch
    (what columns provide the relevant errors and, if available,
     correlations)
\end{itemize}

This usage shows that MANGO must be designed in a way that individual measurements or quantities
can easily be identified as such and manipulated independently of the whole instance.

This document does not recommend one approach over another.
This is a matter for the data providers to decide.

\subsubsection{Xmatch tool }
The basic cross-match of two astronomical tables consists in associating pairs of sources -- one 
from each table -- fulfilling a given angular distance based criterion.
%In relational algebra terms, it is a theta-join on a distance predicate.

More generally, a cross-match is the association of sources from different tables given their 
proximity in an astrometric (but also possibly photometric, statistical, ...) parameter
space \citep{2017A&A...597A..89P} .

If proper motions (plus parallax and radial velocities) are available, the cross-match tool 
may propagate the positions of each table to a common epoch.
It may also take into account positional uncertainties to reject the statistically unlikely associations.

In the latter case (cross-match between two tables taking into account positional errors),
the tool needs to retrieve the errors associated to the each position in each table.

UCDs may help in identifying the errors associated to a positional columns, 
%as shown in table
but this is not sufficient for tables with more complex cases based on multi-parameter cases.
