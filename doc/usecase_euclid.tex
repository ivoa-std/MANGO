Euclid telescope has been designed to unveil some of the questions about the
dark Universe, including dark matter and dark energy, what would include, e.g.
quite accurate measurements of the expansion of the Universe.

Euclid will mainly observe extragalactic objects providing, e.g. information
of the shapes of galaxies, gravitational lensing,  baryon acoustic oscillations
and distances to galaxies using spectroscopic data.

For this mission, and apart from the common metadata provided for extra galactic
sources into astronomical catalogues, a good support for object taxonomy and
shapes of objects will be required. As known due to general relativity effects,
shapes far galaxies could be deformed due to gravitational lensing effects,
producing convergence (visual displacements on the position) and rear (deformation
of the shape) effects. All these metadata should be ready for annotations and,
also, correlated to theoretical or real metadata in other datasets.

Finally, crossmatch information with other catalogues will be of crucial interest
as data from other satellites and, more importantly, from ground based
observatories will be combined with Euclid data to produce consistent scientific
datasets.

A non-exhaustive list of properties required for Euclid use cases would be composed
of:
\begin{itemize}
    \item identifier
    \item sky position
    \item correlation with other catalogues
    \item photometry (ground + satellite )
    \item morphology class
    \item redshift
    \item photometric redshift
\end{itemize}