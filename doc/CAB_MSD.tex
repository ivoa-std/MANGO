\documentclass[11pt,a4paper]{ivoa}
\input tthdefs

\title{A componant based model for source data}

% see ivoatexDoc for what group names to use here
\ivoagroup{DM}


\author{François Bonnarel}
\author{Gilles Landais}
\author{Laurent Michel}
\author{Jesus Salgado}

\editor{Laurent Michel}

% \previousversion[????URL????]{????Concise Document Label????}
\previousversion{This is the first public release}
       

\begin{document}
\begin{abstract}
???? Abstract ????
\end{abstract}


\section*{Acknowledgments}

???? Or remove the section header ????

\section*{Conformance-related definitions}

The words ``MUST'', ``SHALL'', ``SHOULD'', ``MAY'', ``RECOMMENDED'', and
``OPTIONAL'' (in upper or lower case) used in this document are to be
interpreted as described in IETF standard RFC2119 \citep{std:RFC2119}.

The \emph{Virtual Observatory (VO)} is a
general term for a collection of federated resources that can be used
to conduct astronomical research, education, and outreach.
The \href{http://www.ivoa.net}{International
Virtual Observatory Alliance (IVOA)} is a global
collaboration of separately funded projects to develop standards and
infrastructure that enable VO applications.


\section{Introduction}

The source DM is a long term concern for the DM working group and more generally for the IVOA.
In the past years, there were some proposals to design a global model for sources \citep{wd:jesusdm} of for catalogs \citep{wd:catalog}.
Other proposals, more model-agnostic, were focused on the data annotation in VOTables \citep{note:stcvot} \citep{note:seb}. In this case the goal was no longer to design a source model but to provide a complete description of  individual quantities (positions, velocity…).
None of these proposals succeeded for various reasons. 

The source DM issue resurfaced at the spring 2018 Interop in Victoria during an hands-on session focused on the tools available to work with VO data models and especially with VO-DML. The goal of this session was to annotate data from different origins in order to make them interoperable with each other. The main concern expressed during this session was not related to the tools themselves but to the lack of models for sources. 
This is a big paradox in the VO world ; source data which represent the basic bricks of the astronomer work, have no model. This paradox can be explained by the fact that sources data are multifaceted. The way of which source data are organized depends on the survey they come from, one the way they have been generated  and on the expected use. In a more general way, it depends on the science we want to do with them. This diversity cannot be endorsed by a single model. Having a global source model would lead to a very complex solution not usable in practice.

This standard proposes to overcome this paradox presenting model based on independent components and associayted data thata can be embedded on deman in a container.


\subsection{Role within the VO Architecture}

\begin{figure}
\centering

% As of ivoatex 1.2, the architecture diagram is generated by ivoatex in
% SVG; copy ivoatex/archdiag-full.xml to archdiag.xml and throw out
% all lines not relevant to your standard.
% Notes don't generally need this.  If you don't copy archdiag.xml,
% you must remove archdiag.svg from FIGURES in the Makefile.

\includegraphics[width=0.9\textwidth]{role_diagram.pdf}
\caption{Architecture diagram for this document}
\label{fig:archdiag}
\end{figure}

Fig.~\ref{fig:archdiag} shows the role this document plays within the
IVOA architecture \citep{note:VOARCH}.



\section{Use Cases and  Requirements}

\subsection{Use Cases}
\begin{itemize}
    \item archival data    
    \item mission data    
    \item exoplanet data    
    \item crossmatch data   
    \item client data
\end{itemize}



\subsection{Requirements}
\begin{itemize}
    \item identifier, reference position, proper motion, parallax + distance, correlation, source extension, radial velocity (redshift), luminosity, date, multiple detection
   \item identifier, position, Reshift, correlation with Gaia, photometry (ground + sat), morphology, reshift, photometric redshift
   \item Exoplanets: position, orbit, different source level (star, planet, moon), status and classification, orbiting system description
  \item Morphologically Complex Structures
  \item detection (name, pos, time, extension, PHA)
  \item Vizier pre-existing data, grouping columns, lots of available metadata, column name formatting, filter service implemented, one column different frames
All quantities are time dependant, Dependant on calibration + physical model
  \item multiple instances of the same component
\end{itemize}



\section{Model}
\subsection{Overview}
\begin{figure}
\includegraphics[width=0.9\textwidth]{../model/overview_diagram.png}
\caption{Architecture diagram for this document}
\label{fig:archdiag}
\end{figure}



\begin{figure}
\includegraphics[width=0.9\textwidth]{../model/stc_ext_diagram.png}
\caption{Architecture diagram for this document}
\label{fig:archdiag}
\end{figure}


% -------------------------------------------
% Items to substitute into the ivoatex document template.
%
%\ivoagroup{Data Model Working Group}

%\title{Mango}


%\author{Laurent Michel}
    
%\author{Fran??ois Bonnarel}
    
%\author{Gilles Landais}
    
%\author{Mireille Louys}
    
%\author{Marco Molinaro}
    
%\author{Jesue Salgado}
    
%\previousversion{0.0}
      
% -------------------------------------------

\pagebreak
\section{Model: mango }
  
  % INSERT FIGURE HERE
  %\begin{figure}[h]
  %\begin{center}
  %  \includegraphics[width=\textwidth]{????.png}
  %  \caption{???}\label{fig:????}
  %\end{center}
  %\end{figure}

  Data model based oon components and data association for source data

  \subsection{Article}
  \label{sect:Article}
    Reference article for the MANGO source

    \subsubsection{Article.editor}
      \textbf{vodml-id: Article.editor} \newline
      \textbf{type: \hyperref[sect:ivoa]{ivoa:string}} \newline
      \textbf{multiplicity: 1} \newline 
      Article editor name

    \subsubsection{Article.article}
      \textbf{vodml-id: Article.article} \newline
      \textbf{type: \hyperref[sect:ivoa]{ivoa:string}} \newline
      \textbf{multiplicity: 1} \newline 
      Bibcode or DOI of the reference article

  \subsection{AssociatedData (Abstract)}
  \label{sect:AssociatedData}
    Abstract reference to a particular dataset associated to the Source. This class is used to specify the type of the dataset as well as its role.

    \subsubsection{AssociatedData.semantic}
      \textbf{vodml-id: AssociatedData.semantic} \newline
      \textbf{type: \hyperref[sect:VocabularyTerm]{mango:VocabularyTerm}} \newline
      \textbf{multiplicity: 1} \newline 
      Reference to a semantic concept giving the nature of the associated data. As long as the vocabulary is not set, the possible values of this attribute are given by the LinkSemantic enumeration.

    \subsubsection{AssociatedData.description}
      \textbf{vodml-id: AssociatedData.description} \newline
      \textbf{type: \hyperref[sect:ivoa]{ivoa:string}} \newline
      \textbf{multiplicity: 1} \newline 
      Free text description of the associated data

  \subsection{AssociatedMangoInstance}
  \label{sect:AssociatedMangoInstance}
    Reference to another MANGO instance that is part of the associated data.

    \subsubsection{AssociatedMangoInstance.mangoInstance}
      \textbf{vodml-id: AssociatedMangoInstance.mangoInstance} \newline
      \textbf{type: \hyperref[sect:Source]{mango:Source}} \newline
      \textbf{multiplicity: 1} \newline 
      Composition link pointing on one MANGO instance associated with the source.

  \subsection{BitField}
  \label{sect:BitField}
    TODO : Missing description : please, update your UML model asap.

  \subsection{Color}
  \label{sect:Color}
    TODO : Missing description : please, update your UML model asap.

    \subsubsection{Color.colorDef}
      \textbf{vodml-id: Color.colorDef} \newline
      \textbf{type: \hyperref[sect:ColorDef]{mango:ColorDef}} \newline
      \textbf{multiplicity: 1} \newline 
      TODO : Missing description : please, update your UML model asap.

  \subsection{ColorDef}
  \label{sect:ColorDef}
    TODO : Missing description : please, update your UML model asap.

    \subsubsection{ColorDef.definition}
      \textbf{vodml-id: ColorDef.definition} \newline
      \textbf{type: \hyperref[sect:ColorDefinition]{mango:ColorDefinition}} \newline
      \textbf{multiplicity: 1} \newline 
      TODO : Missing description : please, update your UML model asap.

    \subsubsection{ColorDef.high}
      \textbf{vodml-id: ColorDef.high} \newline
      \textbf{type: :PhotFilter} \newline
      \textbf{multiplicity: 1} \newline 
      TODO : Missing description : please, update your UML model asap.

    \subsubsection{ColorDef.low}
      \textbf{vodml-id: ColorDef.low} \newline
      \textbf{type: :PhotFilter} \newline
      \textbf{multiplicity: 1} \newline 
      TODO : Missing description : please, update your UML model asap.

  \subsection{Correlation11}
  \label{sect:Correlation11}
    Error of a 1D property (A) correlated with a 1D parameter (B): $error\\_A = correlation\\_A1B1 * value\\_B$

    \subsubsection{Correlation11.a1b1}
      \textbf{vodml-id: Correlation11.a1b1} \newline
      \textbf{type: \hyperref[sect:ivoa]{ivoa:real}} \newline
      \textbf{multiplicity: 1} \newline 
      Correlation coefficient giving the contribution of the first axis of \texttt{B} to the error on the first axis of \texttt{A}.

  \subsection{Correlation12}
  \label{sect:Correlation12}
    Error of a 1D property (A) correlated with a 2D parameter (B): $error\\_A = correlation\\_A1B1 * value\\_B\_1 + correlation\\_A1B2 * value\\_B\_2$

    \subsubsection{Correlation12.a1b1}
      \textbf{vodml-id: Correlation12.a1b1} \newline
      \textbf{type: \hyperref[sect:ivoa]{ivoa:real}} \newline
      \textbf{multiplicity: 1} \newline 
      Correlation coefficient giving the contribution of the first axis of \texttt{B} to the error on the first axis of \texttt{A}.

    \subsubsection{Correlation12.a1b2}
      \textbf{vodml-id: Correlation12.a1b2} \newline
      \textbf{type: \hyperref[sect:ivoa]{ivoa:real}} \newline
      \textbf{multiplicity: 1} \newline 
      Correlation coefficient giving the contribution of the second axis of \texttt{B} to the error on the first axis of \texttt{A}.

  \subsection{Correlation21}
  \label{sect:Correlation21}
    Error of a 2D property (A) correlated with a 1D parameter (B): $error\\_A\_1 = correlation\\_A1B1 * value\\_B$ $error\\_A\_2 = correlation\\_A12B1 * value\\_B$

    \subsubsection{Correlation21.a2b1}
      \textbf{vodml-id: Correlation21.a2b1} \newline
      \textbf{type: \hyperref[sect:ivoa]{ivoa:real}} \newline
      \textbf{multiplicity: 1} \newline 
      Correlation coefficient giving the contribution of the second axis of \texttt{B} to the error on the first axis of \texttt{A}.

    \subsubsection{Correlation21.a1b1}
      \textbf{vodml-id: Correlation21.a1b1} \newline
      \textbf{type: \hyperref[sect:ivoa]{ivoa:real}} \newline
      \textbf{multiplicity: 1} \newline 
      Correlation coefficient giving the contribution of the first axis of \texttt{B} to the error on the first axis of \texttt{A}.

  \subsection{Correlation22}
  \label{sect:Correlation22}
    Error of a 2D property (A) correlated with a 2D parameter (B): $error\\_A\_1 = correlation\\_A1B1 * value\\_B\_1 + correlation\\_A1B2 * value\\_B\_2$ $error\\_A\_2 = correlation\\_A2B1 * value\\_B\_1 + correlation\\_A2B2 * value\\_B\_2$

    \subsubsection{Correlation22.a2b1}
      \textbf{vodml-id: Correlation22.a2b1} \newline
      \textbf{type: \hyperref[sect:ivoa]{ivoa:real}} \newline
      \textbf{multiplicity: 1} \newline 
      Correlation coefficient giving the contribution of the first axis of \texttt{B} to the error on the second axis of \texttt{A}.

    \subsubsection{Correlation22.a2b2}
      \textbf{vodml-id: Correlation22.a2b2} \newline
      \textbf{type: \hyperref[sect:ivoa]{ivoa:real}} \newline
      \textbf{multiplicity: 1} \newline 
      Correlation coefficient giving the contribution of the second axis of \texttt{B} to the error on the second axis of \texttt{A}.

    \subsubsection{Correlation22.a1b1}
      \textbf{vodml-id: Correlation22.a1b1} \newline
      \textbf{type: \hyperref[sect:ivoa]{ivoa:real}} \newline
      \textbf{multiplicity: 1} \newline 
      Correlation coefficient giving the contribution of the first axis of \texttt{B} to the error on the first axis of \texttt{A}.

    \subsubsection{Correlation22.a1b2}
      \textbf{vodml-id: Correlation22.a1b2} \newline
      \textbf{type: \hyperref[sect:ivoa]{ivoa:real}} \newline
      \textbf{multiplicity: 1} \newline 
      Correlation coefficient giving the contribution of the second axis of \texttt{B} to the error on the first axis of \texttt{A}.

  \subsection{DataOrigin}
  \label{sect:DataOrigin}
    Class representing the origin of the data following the DCP note (TBD)

    \subsubsection{DataOrigin.citation}
      \textbf{vodml-id: DataOrigin.citation} \newline
      \textbf{type: \hyperref[sect:ivoa]{ivoa:string}} \newline
      \textbf{multiplicity: 1} \newline 
      Dataset idnetifier that can be used for citation (e.g. DOI)

    \subsubsection{DataOrigin.reference_url}
      \textbf{vodml-id: DataOrigin.reference_url} \newline
      \textbf{type: \hyperref[sect:ivoa]{ivoa:string}} \newline
      \textbf{multiplicity: 1} \newline 
      Dataset landing page

    \subsubsection{DataOrigin.resource_version}
      \textbf{vodml-id: DataOrigin.resource_version} \newline
      \textbf{type: \hyperref[sect:ivoa]{ivoa:string}} \newline
      \textbf{multiplicity: 1} \newline 
      Dataset version

    \subsubsection{DataOrigin.creator}
      \textbf{vodml-id: DataOrigin.creator} \newline
      \textbf{type: \hyperref[sect:ivoa]{ivoa:string}} \newline
      \textbf{multiplicity: 1} \newline 
      Person(s) mainly involved in the creation of the resource, generally the author.

    \subsubsection{DataOrigin.cites}
      \textbf{vodml-id: DataOrigin.cites} \newline
      \textbf{type: \hyperref[sect:ivoa]{ivoa:string}} \newline
      \textbf{multiplicity: 1} \newline 
      Identifier (IVOID, DOI or Bibcode) of a second resource using relation of type \texttt{cites} (\url{https://www.ivoa.net/rdf/voresource/relationship\_type/})

    \subsubsection{DataOrigin.is_derived_from}
      \textbf{vodml-id: DataOrigin.is_derived_from} \newline
      \textbf{type: \hyperref[sect:ivoa]{ivoa:string}} \newline
      \textbf{multiplicity: 1} \newline 
      Identifier (IVOID, DOI or Bibcode) of a second resource using relation of type \texttt{is\_derived\_from} (\url{https://www.ivoa.net/rdf/voresource/relationship\_type/})

    \subsubsection{DataOrigin.original_date}
      \textbf{vodml-id: DataOrigin.original_date} \newline
      \textbf{type: \hyperref[sect:ivoa]{ivoa:datetime}} \newline
      \textbf{multiplicity: 1} \newline 
      Date of the original resource from which the MANGO source instance is derived

    \subsubsection{DataOrigin.query}
      \textbf{vodml-id: DataOrigin.query} \newline
      \textbf{type: \hyperref[sect:QueryOrigin]{mango:QueryOrigin}} \newline
      \textbf{multiplicity: 0..1} \newline 
      TODO : Missing description : please, update your UML model asap.

    \subsubsection{DataOrigin.rights}
      \textbf{vodml-id: DataOrigin.rights} \newline
      \textbf{type: \hyperref[sect:License]{mango:License}} \newline
      \textbf{multiplicity: 0..1} \newline 
      TODO : Missing description : please, update your UML model asap.

    \subsubsection{DataOrigin.article}
      \textbf{vodml-id: DataOrigin.article} \newline
      \textbf{type: \hyperref[sect:Article]{mango:Article}} \newline
      \textbf{multiplicity: 0..1} \newline 
      TODO : Missing description : please, update your UML model asap.

  \subsection{EpochPosition}
  \label{sect:EpochPosition}
    Class grouping all parameters needed to define an object position at a given epoch. The space coordinate system is common to all attributes to insure the consistance between all of the instance parameters.

    \subsubsection{EpochPosition.longitude}
      \textbf{vodml-id: EpochPosition.longitude} \newline
      \textbf{type: \hyperref[sect:ivoa]{ivoa:RealQuantity}} \newline
      \textbf{multiplicity: 1} \newline 
      TODO : Missing description : please, update your UML model asap.

    \subsubsection{EpochPosition.latitude}
      \textbf{vodml-id: EpochPosition.latitude} \newline
      \textbf{type: \hyperref[sect:ivoa]{ivoa:RealQuantity}} \newline
      \textbf{multiplicity: 1} \newline 
      TODO : Missing description : please, update your UML model asap.

    \subsubsection{EpochPosition.parallax}
      \textbf{vodml-id: EpochPosition.parallax} \newline
      \textbf{type: \hyperref[sect:ivoa]{ivoa:RealQuantity}} \newline
      \textbf{multiplicity: 1} \newline 
      TODO : Missing description : please, update your UML model asap.

    \subsubsection{EpochPosition.radialVelocity}
      \textbf{vodml-id: EpochPosition.radialVelocity} \newline
      \textbf{type: \hyperref[sect:ivoa]{ivoa:RealQuantity}} \newline
      \textbf{multiplicity: 1} \newline 
      TODO : Missing description : please, update your UML model asap.

    \subsubsection{EpochPosition.pmLongitude}
      \textbf{vodml-id: EpochPosition.pmLongitude} \newline
      \textbf{type: \hyperref[sect:ivoa]{ivoa:RealQuantity}} \newline
      \textbf{multiplicity: 1} \newline 
      TODO : Missing description : please, update your UML model asap.

    \subsubsection{EpochPosition.pmLatitude}
      \textbf{vodml-id: EpochPosition.pmLatitude} \newline
      \textbf{type: \hyperref[sect:ivoa]{ivoa:RealQuantity}} \newline
      \textbf{multiplicity: 1} \newline 
      TODO : Missing description : please, update your UML model asap.

    \subsubsection{EpochPosition.epoch}
      \textbf{vodml-id: EpochPosition.epoch} \newline
      \textbf{type: coords:Epoch} \newline
      \textbf{multiplicity: 1} \newline 
      TODO : Missing description : please, update your UML model asap.

    \subsubsection{EpochPosition.pmCosDeltApplied}
      \textbf{vodml-id: EpochPosition.pmCosDeltApplied} \newline
      \textbf{type: \hyperref[sect:ivoa]{ivoa:boolean}} \newline
      \textbf{multiplicity: 1} \newline 
      TODO : Missing description : please, update your UML model asap.

    \subsubsection{EpochPosition.errors}
      \textbf{vodml-id: EpochPosition.errors} \newline
      \textbf{type: \hyperref[sect:EpochPositionErrors]{mango:EpochPositionErrors}} \newline
      \textbf{multiplicity: 0..1} \newline 
      TODO : Missing description : please, update your UML model asap.

    \subsubsection{EpochPosition.correlations}
      \textbf{vodml-id: EpochPosition.correlations} \newline
      \textbf{type: \hyperref[sect:EpochPositionCorrelations]{mango:EpochPositionCorrelations}} \newline
      \textbf{multiplicity: 0..1} \newline 
      TODO : Missing description : please, update your UML model asap.

    \subsubsection{EpochPosition.coordSys}
      \textbf{vodml-id: EpochPosition.coordSys} \newline
      \textbf{type: coords:SpaceSys} \newline
      \textbf{multiplicity: 0..1} \newline 
      TODO : Missing description : please, update your UML model asap.

  \subsection{EpochPositionCorrelations}
  \label{sect:EpochPositionCorrelations}
    TODO : Missing description : please, update your UML model asap.

    \subsubsection{EpochPositionCorrelations.positionPm}
      \textbf{vodml-id: EpochPositionCorrelations.positionPm} \newline
      \textbf{type: \hyperref[sect:Correlation22]{mango:Correlation22}} \newline
      \textbf{multiplicity: 0..1} \newline 
      Correlated error between the proper motion and the position. This error is represented by a 2D matrix. $error\_{propermotion} = f(position)$

    \subsubsection{EpochPositionCorrelations.parallaxPm}
      \textbf{vodml-id: EpochPositionCorrelations.parallaxPm} \newline
      \textbf{type: \hyperref[sect:Correlation12]{mango:Correlation12}} \newline
      \textbf{multiplicity: 0..1} \newline 
      Correlated error between the parallax and the proper motion. The parallax error might depend on the proper motion vector. $error\_{parallax} = f(properMotion)$

    \subsubsection{EpochPositionCorrelations.positionParallax}
      \textbf{vodml-id: EpochPositionCorrelations.positionParallax} \newline
      \textbf{type: \hyperref[sect:Correlation21]{mango:Correlation21}} \newline
      \textbf{multiplicity: 0..1} \newline 
      Correlated error between the parallax and the position. The parallax error might depend on the sky position. $error\_{parallax} = f(position)$

    \subsubsection{EpochPositionCorrelations.positionPosition}
      \textbf{vodml-id: EpochPositionCorrelations.positionPosition} \newline
      \textbf{type: \hyperref[sect:Correlation22]{mango:Correlation22}} \newline
      \textbf{multiplicity: 0..1} \newline 
      TODO : Missing description : please, update your UML model asap.

    \subsubsection{EpochPositionCorrelations.properMotionPm}
      \textbf{vodml-id: EpochPositionCorrelations.properMotionPm} \newline
      \textbf{type: \hyperref[sect:Correlation22]{mango:Correlation22}} \newline
      \textbf{multiplicity: 0..1} \newline 
      TODO : Missing description : please, update your UML model asap.

  \subsection{EpochPositionErrors}
  \label{sect:EpochPositionErrors}
    Class for the error attached to a EpochPosition. The component in this class represent the errors of individual parameters as well as errors due to the parameter correlations.

    \subsubsection{EpochPositionErrors.parallax}
      \textbf{vodml-id: EpochPositionErrors.parallax} \newline
      \textbf{type: \hyperref[sect:ErrorTypes.PropertyError1D]{mango:ErrorTypes.PropertyError1D}} \newline
      \textbf{multiplicity: 0..1} \newline 
      Parallax error. This error is meant to be symetrical.

    \subsubsection{EpochPositionErrors.radialVelocity}
      \textbf{vodml-id: EpochPositionErrors.radialVelocity} \newline
      \textbf{type: \hyperref[sect:ErrorTypes.PropertyError1D]{mango:ErrorTypes.PropertyError1D}} \newline
      \textbf{multiplicity: 0..1} \newline 
      Error in the radial velocity. This error is meant to be symetrical.

    \subsubsection{EpochPositionErrors.position}
      \textbf{vodml-id: EpochPositionErrors.position} \newline
      \textbf{type: \hyperref[sect:ErrorTypes.PropertyError2D]{mango:ErrorTypes.PropertyError2D}} \newline
      \textbf{multiplicity: 0..1} \newline 
      TODO : Missing description : please, update your UML model asap.

    \subsubsection{EpochPositionErrors.properMotion}
      \textbf{vodml-id: EpochPositionErrors.properMotion} \newline
      \textbf{type: \hyperref[sect:ErrorTypes.PropertyError2D]{mango:ErrorTypes.PropertyError2D}} \newline
      \textbf{multiplicity: 0..1} \newline 
      TODO : Missing description : please, update your UML model asap.

  \subsection{Label}
  \label{sect:Label}
    TODO : Missing description : please, update your UML model asap.

    \subsubsection{Label.text}
      \textbf{vodml-id: Label.text} \newline
      \textbf{type: \hyperref[sect:ivoa]{ivoa:string}} \newline
      \textbf{multiplicity: 1} \newline 
      TODO : Missing description : please, update your UML model asap.

  \subsection{License}
  \label{sect:License}
    Place holder for the license covering the MANGO instance

    \subsubsection{License.rights_uri}
      \textbf{vodml-id: License.rights_uri} \newline
      \textbf{type: \hyperref[sect:ivoa]{ivoa:string}} \newline
      \textbf{multiplicity: 1} \newline 
      Licence URI following the regostry practice. This should come from SPDX \url{https://spdx.org/licenses}. Creatives Commons (\url{https://creativecommons.org}) are also accepted.

    \subsubsection{License.rights}
      \textbf{vodml-id: License.rights} \newline
      \textbf{type: \hyperref[sect:ivoa]{ivoa:string}} \newline
      \textbf{multiplicity: 1} \newline 
      License or Copyright text

  \subsection{PhotometricProperty}
  \label{sect:PhotometricProperty}
    TODO : Missing description : please, update your UML model asap.

    \subsubsection{PhotometricProperty.value}
      \textbf{vodml-id: PhotometricProperty.value} \newline
      \textbf{type: \hyperref[sect:ivoa]{ivoa:RealQuantity}} \newline
      \textbf{multiplicity: 1} \newline 
      TODO : Missing description : please, update your UML model asap.

    \subsubsection{PhotometricProperty.photCal}
      \textbf{vodml-id: PhotometricProperty.photCal} \newline
      \textbf{type: :PhotCal} \newline
      \textbf{multiplicity: 1} \newline 
      TODO : Missing description : please, update your UML model asap.

  \subsection{PhysicalProperty}
  \label{sect:PhysicalProperty}
    TODO : Missing description : please, update your UML model asap.

    \subsubsection{PhysicalProperty.calibrationLevel}
      \textbf{vodml-id: PhysicalProperty.calibrationLevel} \newline
      \textbf{type: \hyperref[sect:CalibrationLevel]{mango:CalibrationLevel}} \newline
      \textbf{multiplicity: 1} \newline 
      TODO : Missing description : please, update your UML model asap.

    \subsubsection{PhysicalProperty.measure}
      \textbf{vodml-id: PhysicalProperty.measure} \newline
      \textbf{type: meas:Measure} \newline
      \textbf{multiplicity: 1} \newline 
      TODO : Missing description : please, update your UML model asap.

  \subsection{Property}
  \label{sect:Property}
    Reference to a particular measure of the Source. This class is used to specify the type of the measure as well as its role.

    \noindent \textbf{constraint} \newline
    \indent    \textbf{detail: Property.One association at the time
 }\newline


    \subsubsection{Property.semantic}
      \textbf{vodml-id: Property.semantic} \newline
      \textbf{type: \hyperref[sect:VocabularyTerm]{mango:VocabularyTerm}} \newline
      \textbf{multiplicity: 1} \newline 
      Reference to a semantic concept giving the nature of the parameter As long as the vocabulary is not set, the possible values of this attribute are given by the ParamSemantic enumeration.

    \subsubsection{Property.description}
      \textbf{vodml-id: Property.description} \newline
      \textbf{type: \hyperref[sect:ivoa]{ivoa:string}} \newline
      \textbf{multiplicity: 1} \newline 
      Free text description of the measure

    \subsubsection{Property.measure}
      \textbf{vodml-id: Property.measure} \newline
      \textbf{type: meas:Measure} \newline
      \textbf{multiplicity: 1} \newline 
      Composition link pointing to the meas:Measure instance

    \subsubsection{Property.associatedProperties}
      \textbf{vodml-id: Property.associatedProperties} \newline
      \textbf{type: \hyperref[sect:Property]{mango:Property}} \newline
      \textbf{multiplicity: 1..*} \newline 
      <Enter note text here>

  \subsection{QuantityCorrelation (Abstract)}
  \label{sect:QuantityCorrelation}
    Abstract root class for the correlated errors. In the subclasses nomenclature the error relates to a property named \texttt{A} in correlation with a property named {B).

    \subsubsection{QuantityCorrelation.isCovariance}
      \textbf{vodml-id: QuantityCorrelation.isCovariance} \newline
      \textbf{type: \hyperref[sect:ivoa]{ivoa:boolean}} \newline
      \textbf{multiplicity: 0..1} \newline 
      Optional parameter giving the UCD of the correlated property. This attribute may facilitate the interpretation of complex error patterns.

  \subsection{QueryOrigin}
  \label{sect:QueryOrigin}
    TODO : Missing description : please, update your UML model asap.

    \subsubsection{QueryOrigin.ivoid}
      \textbf{vodml-id: QueryOrigin.ivoid} \newline
      \textbf{type: \hyperref[sect:ivoa]{ivoa:string}} \newline
      \textbf{multiplicity: 1} \newline 
      IVOID of the underlying data collection

    \subsubsection{QueryOrigin.publisher}
      \textbf{vodml-id: QueryOrigin.publisher} \newline
      \textbf{type: \hyperref[sect:ivoa]{ivoa:string}} \newline
      \textbf{multiplicity: 1} \newline 
      Data center that produced the MANGOI source instance

    \subsubsection{QueryOrigin.server_software}
      \textbf{vodml-id: QueryOrigin.server_software} \newline
      \textbf{type: \hyperref[sect:ivoa]{ivoa:string}} \newline
      \textbf{multiplicity: 1} \newline 
      Version of the software the produced the MANO source instance. It is encouraged to follow \url{https://ivoa.net/documents/Notes/softid/index.html}.

    \subsubsection{QueryOrigin.service_protocol}
      \textbf{vodml-id: QueryOrigin.service_protocol} \newline
      \textbf{type: \hyperref[sect:ivoa]{ivoa:string}} \newline
      \textbf{multiplicity: 1} \newline 
      IVOID of the protocol through which the data was retrieved

    \subsubsection{QueryOrigin.request}
      \textbf{vodml-id: QueryOrigin.request} \newline
      \textbf{type: \hyperref[sect:ivoa]{ivoa:string}} \newline
      \textbf{multiplicity: 1} \newline 
      Full request URL including a query string. For the simple protocols,put the url-encoded form of the query parameters. For TAP queries, use the /sync UWS URL. The format is free for others request types.

    \subsubsection{QueryOrigin.query}
      \textbf{vodml-id: QueryOrigin.query} \newline
      \textbf{type: \hyperref[sect:ivoa]{ivoa:string}} \newline
      \textbf{multiplicity: 1} \newline 
      Input query in a formal langage such as ADQL.

    \subsubsection{QueryOrigin.request_date}
      \textbf{vodml-id: QueryOrigin.request_date} \newline
      \textbf{type: \hyperref[sect:ivoa]{ivoa:datetime}} \newline
      \textbf{multiplicity: 1} \newline 
      Query execution date

    \subsubsection{QueryOrigin.contact}
      \textbf{vodml-id: QueryOrigin.contact} \newline
      \textbf{type: \hyperref[sect:ivoa]{ivoa:string}} \newline
      \textbf{multiplicity: 1} \newline 
      Email or URL to contact the publisher

  \subsection{SatusValues}
  \label{sect:SatusValues}
    TODO : Missing description : please, update your UML model asap.

    \subsubsection{SatusValues.values}
      \textbf{vodml-id: SatusValues.values} \newline
      \textbf{type: \hyperref[sect:StatusValue]{mango:StatusValue}} \newline
      \textbf{multiplicity: 1..*} \newline 
      TODO : Missing description : please, update your UML model asap.

  \subsection{Shape}
  \label{sect:Shape}
    TODO : Missing description : please, update your UML model asap.

    \subsubsection{Shape.shape}
      \textbf{vodml-id: Shape.shape} \newline
      \textbf{type: \hyperref[sect:ivoa]{ivoa:string}} \newline
      \textbf{multiplicity: 1} \newline 
      TODO : Missing description : please, update your UML model asap.

    \subsubsection{Shape.serialization}
      \textbf{vodml-id: Shape.serialization} \newline
      \textbf{type: \hyperref[sect:ShapeSerialization]{mango:ShapeSerialization}} \newline
      \textbf{multiplicity: 1} \newline 
      TODO : Missing description : please, update your UML model asap.

    \subsubsection{Shape.coordSys}
      \textbf{vodml-id: Shape.coordSys} \newline
      \textbf{type: coords:SpaceSys} \newline
      \textbf{multiplicity: 0..1} \newline 
      TODO : Missing description : please, update your UML model asap.

  \subsection{Source}
  \label{sect:Source}
    Root class of the model. MANGO instances are meant of be Source instances. A source is something with an identifier and two docks: one for the parameters and one for the associated data.

    \subsubsection{Source.identifier}
      \textbf{vodml-id: Source.identifier} \newline
      \textbf{type: \hyperref[sect:ivoa]{ivoa:string}} \newline
      \textbf{multiplicity: 1} \newline 
      Unique identifier for a Source. The uniqueness of that identifier is not managed by the model. The format is free.

    \subsubsection{Source.associatedDataDock}
      \textbf{vodml-id: Source.associatedDataDock} \newline
      \textbf{type: \hyperref[sect:AssociatedData]{mango:AssociatedData}} \newline
      \textbf{multiplicity: 0..*} \newline 
      Composition link pointing on all data associated with the source.

    \subsubsection{Source.propertyDock}
      \textbf{vodml-id: Source.propertyDock} \newline
      \textbf{type: \hyperref[sect:Property]{mango:Property}} \newline
      \textbf{multiplicity: 0..*} \newline 
      Composition link pointing on all parameters attached to the source.

    \subsubsection{Source.dataOrigin}
      \textbf{vodml-id: Source.dataOrigin} \newline
      \textbf{type: \hyperref[sect:DataOrigin]{mango:DataOrigin}} \newline
      \textbf{multiplicity: 0..1} \newline 
      TODO : Missing description : please, update your UML model asap.

  \subsection{Status}
  \label{sect:Status}
    TODO : Missing description : please, update your UML model asap.

    \subsubsection{Status.status}
      \textbf{vodml-id: Status.status} \newline
      \textbf{type: \hyperref[sect:ivoa]{ivoa:integer}} \newline
      \textbf{multiplicity: 1} \newline 
      TODO : Missing description : please, update your UML model asap.

    \subsubsection{Status.allowedValues}
      \textbf{vodml-id: Status.allowedValues} \newline
      \textbf{type: \hyperref[sect:SatusValues]{mango:SatusValues}} \newline
      \textbf{multiplicity: 0..1} \newline 
      TODO : Missing description : please, update your UML model asap.

  \subsection{StatusValue}
  \label{sect:StatusValue}
    TODO : Missing description : please, update your UML model asap.

    \subsubsection{StatusValue.value}
      \textbf{vodml-id: StatusValue.value} \newline
      \textbf{type: \hyperref[sect:ivoa]{ivoa:integer}} \newline
      \textbf{multiplicity: 1} \newline 
      TODO : Missing description : please, update your UML model asap.

    \subsubsection{StatusValue.description}
      \textbf{vodml-id: StatusValue.description} \newline
      \textbf{type: \hyperref[sect:ivoa]{ivoa:string}} \newline
      \textbf{multiplicity: 1} \newline 
      TODO : Missing description : please, update your UML model asap.

  \subsection{VocabularyTerm}
  \label{sect:VocabularyTerm}
    TODO : Missing description : please, update your UML model asap.

    \subsubsection{VocabularyTerm.uri}
      \textbf{vodml-id: VocabularyTerm.uri} \newline
      \textbf{type: \hyperref[sect:ivoa]{ivoa:string}} \newline
      \textbf{multiplicity: 1} \newline 
      TODO : Missing description : please, update your UML model asap.

    \subsubsection{VocabularyTerm.label}
      \textbf{vodml-id: VocabularyTerm.label} \newline
      \textbf{type: \hyperref[sect:ivoa]{ivoa:string}} \newline
      \textbf{multiplicity: 1} \newline 
      TODO : Missing description : please, update your UML model asap.

  \subsection{WebEndpoint}
  \label{sect:WebEndpoint}
    Class for associated data referenced by an URL

    \subsubsection{WebEndpoint.ContentType}
      \textbf{vodml-id: WebEndpoint.ContentType} \newline
      \textbf{type: \hyperref[sect:ivoa]{ivoa:string}} \newline
      \textbf{multiplicity: 1} \newline 
      Mime type of the URL

    \subsubsection{WebEndpoint.url}
      \textbf{vodml-id: WebEndpoint.url} \newline
      \textbf{type: \hyperref[sect:ivoa]{ivoa:anyURI}} \newline
      \textbf{multiplicity: 1} \newline 
      Web endpoint

  \subsection{ShapeFrame}
  \label{sect:ShapeFrame}

  Possible options to encode a shape in a string.

  \noindent \underline{Enumeration Literals}
  \vspace{-\parsep}
  \small
  \begin{itemize}
  
    \item[\textbf{STC\_S}]: \textbf{vodml-id:} ShapeFrame.STC\_S \newline
          \textbf{description:} STCs serialisation
    \item[\textbf{MOC}]: \textbf{vodml-id:} ShapeFrame.MOC \newline
          \textbf{description:} MOC serialisation
  \end{itemize}
  \normalsize


  \subsection{ShapeSerialization}
  \label{sect:ShapeSerialization}

  TODO : Missing description : please, update your UML model asap.

  \noindent \underline{Enumeration Literals}
  \vspace{-\parsep}
  \small
  \begin{itemize}
  
    \item[\textbf{MOC}]: \textbf{vodml-id:} ShapeSerialization.MOC \newline
          \textbf{description:} TODO : Missing description : please, update your UML model asap.
    \item[\textbf{STCS}]: \textbf{vodml-id:} ShapeSerialization.STCS \newline
          \textbf{description:} TODO : Missing description : please, update your UML model asap.
    \item[\textbf{POLYGON}]: \textbf{vodml-id:} ShapeSerialization.POLYGON \newline
          \textbf{description:} TODO : Missing description : please, update your UML model asap.
  \end{itemize}
  \normalsize


  \subsection{CalibrationLevel}
  \label{sect:CalibrationLevel}

  Science ready data with the instrument signature removed (ObsCore).

  \noindent \underline{Enumeration Literals}
  \vspace{-\parsep}
  \small
  \begin{itemize}
  
    \item[\textbf{Raw}]: \textbf{vodml-id:} CalibrationLevel.Raw \newline
          \textbf{description:} Raw instrumental data, in a proprietary or internal data provider defined format, that needs instrument specific tools to be handled (ObsCore).
    \item[\textbf{Instrumental}]: \textbf{vodml-id:} CalibrationLevel.Instrumental \newline
          \textbf{description:} Instrumental data in a standard format which could be manipulated with standard astronomical packages (ObsCore).
    \item[\textbf{Calibrated}]: \textbf{vodml-id:} CalibrationLevel.Calibrated \newline
          \textbf{description:} Science ready data with the instrument signature removed (ObsCore).
  \end{itemize}
  \normalsize


  \subsection{ColorDefinition}
  \label{sect:ColorDefinition}

  TODO : Missing description : please, update your UML model asap.

  \noindent \underline{Enumeration Literals}
  \vspace{-\parsep}
  \small
  \begin{itemize}
  
    \item[\textbf{ColorIndex}]: \textbf{vodml-id:} ColorDefinition.ColorIndex \newline
          \textbf{description:} Difference of mangnitudes: typically $M\_B - M\_v$
    \item[\textbf{HardnessRatio}]: \textbf{vodml-id:} ColorDefinition.HardnessRatio \newline
          \textbf{description:} Normalized ratio of fluxes: $(F\_{EB2} - F\_{EB1}) / (F\_{EB2} + F\_{EB1})$
  \end{itemize}
  \normalsize


\pagebreak
\section{Package: ErrorTypes }

  % INSERT FIGURE HERE
  %\begin{figure}[h]
  %\begin{center}
  %  \includegraphics[width=\textwidth]{????.png}
  %  \caption{???}\label{fig:????}
  %\end{center}
  %\end{figure}

  TODO : Missing description : please, update your UML model asap.

  \subsection{DiagElems2x2}
  \label{sect:ErrorTypes.DiagElems2x2}
    Datatype containing the 2 diagonal elements of a 2x2 matrix. Attributes are named $\sigma$ because this datatype is mostly used in the context of complex errors.

    \subsubsection{DiagElems2x2.sigma1}
      \textbf{vodml-id: ErrorTypes.DiagElems2x2.sigma1} \newline
      \textbf{type: \hyperref[sect:ivoa]{ivoa:real}} \newline
      \textbf{multiplicity: 1} \newline 
      Variance on the first axis.

    \subsubsection{DiagElems2x2.sigma2}
      \textbf{vodml-id: ErrorTypes.DiagElems2x2.sigma2} \newline
      \textbf{type: \hyperref[sect:ivoa]{ivoa:real}} \newline
      \textbf{multiplicity: 1} \newline 
      Variance on the second axis

  \subsection{Ellipse}
  \label{sect:ErrorTypes.Ellipse}
    Elliptical errror for 2D parameters such as the sky positions.

    \subsubsection{Ellipse.semiMajorAxis}
      \textbf{vodml-id: ErrorTypes.Ellipse.semiMajorAxis} \newline
      \textbf{type: \hyperref[sect:ivoa]{ivoa:real}} \newline
      \textbf{multiplicity: 1} \newline 
      Half of the ellipse major axis.

    \subsubsection{Ellipse.semiMinorAxis}
      \textbf{vodml-id: ErrorTypes.Ellipse.semiMinorAxis} \newline
      \textbf{type: \hyperref[sect:ivoa]{ivoa:real}} \newline
      \textbf{multiplicity: 1} \newline 
      Half of the ellipse minor axis.

    \subsubsection{Ellipse.angle}
      \textbf{vodml-id: ErrorTypes.Ellipse.angle} \newline
      \textbf{type: \hyperref[sect:ivoa]{ivoa:RealQuantity}} \newline
      \textbf{multiplicity: 1} \newline 
      Angle between the North Polar Cape (NCP) and the major axis. This angle must be positive taking into account that angles are positive from North to the East. The angle has its own unit.

  \subsection{ErrorMatrix}
  \label{sect:ErrorTypes.ErrorMatrix}
    Diagonal 2D matrix. Non diagonal elements are null.

    \subsubsection{ErrorMatrix.sigma1}
      \textbf{vodml-id: ErrorTypes.ErrorMatrix.sigma1} \newline
      \textbf{type: \hyperref[sect:ivoa]{ivoa:real}} \newline
      \textbf{multiplicity: 1} \newline 
      First diagonal element ($(x\_11)$)

    \subsubsection{ErrorMatrix.sigma2}
      \textbf{vodml-id: ErrorTypes.ErrorMatrix.sigma2} \newline
      \textbf{type: \hyperref[sect:ivoa]{ivoa:real}} \newline
      \textbf{multiplicity: 1} \newline 
      Second diagonal element ($(x\_22)$)

    \subsubsection{ErrorMatrix.covariance}
      \textbf{vodml-id: ErrorTypes.ErrorMatrix.covariance} \newline
      \textbf{type: \hyperref[sect:ivoa]{ivoa:real}} \newline
      \textbf{multiplicity: 1} \newline 
      TODO : Missing description : please, update your UML model asap.

    \subsubsection{ErrorMatrix.isCovariance}
      \textbf{vodml-id: ErrorTypes.ErrorMatrix.isCovariance} \newline
      \textbf{type: \hyperref[sect:ivoa]{ivoa:boolean}} \newline
      \textbf{multiplicity: 1} \newline 
      TODO : Missing description : please, update your UML model asap.

  \subsection{PropertyError (Abstract)}
  \label{sect:ErrorTypes.PropertyError}
    Root (abstract) class of the errors that can be attached to a MA?GO property. The class inherit from \texttt{meas:uncertainty} in order to be usable in the context of properties based on \texttt{Measures} classes.

    \subsubsection{PropertyError.confidenceLevel}
      \textbf{vodml-id: ErrorTypes.PropertyError.confidenceLevel} \newline
      \textbf{type: \hyperref[sect:ivoa]{ivoa:integer}} \newline
      \textbf{multiplicity: 1} \newline 
      Confidence level of the error, expressed in $\sigma$.

    \subsubsection{PropertyError.unit}
      \textbf{vodml-id: ErrorTypes.PropertyError.unit} \newline
      \textbf{type: \hyperref[sect:ivoa]{ivoa:Unit}} \newline
      \textbf{multiplicity: 1} \newline 
      Error unit. It must be compliant with the \texttt{VOUnit} standard. The error unit must be consistant with the unit of the property the error is attached with. This is not checked at the model level.

  \subsection{PropertyError1D}
  \label{sect:ErrorTypes.PropertyError1D}
    Symetrical error for 1D parameter.

    \subsubsection{PropertyError1D.sigma}
      \textbf{vodml-id: ErrorTypes.PropertyError1D.sigma} \newline
      \textbf{type: \hyperref[sect:ivoa]{ivoa:real}} \newline
      \textbf{multiplicity: 1} \newline 
      Error amplitude.

  \subsection{PropertyError2D (Abstract)}
  \label{sect:ErrorTypes.PropertyError2D}
    TODO : Missing description : please, update your UML model asap.

\appendix
\section{Changes from Previous Versions}

No previous versions yet.  
% these would be subsections "Changes from v. WD-..."
% Use itemize environments.


% NOTE: IVOA recommendations must be cited from docrepo rather than ivoabib
% (REC entries there are for legacy documents only)
\bibliography{myrefs,ivoatex/ivoabib,ivoatex/docrepo}


\end{document}
